% Created 2009-12-14 Mon 20:02
\documentclass[11pt]{article}
\usepackage[utf8]{inputenc}
\usepackage[T1]{fontenc}
\usepackage{graphicx}
\usepackage{longtable}
\usepackage{hyperref}


\title{manual}
\author{Alex Sparrow}
\date{14 December 2009}

\begin{document}

\maketitle

\setcounter{tocdepth}{3}
\tableofcontents
\vspace*{1cm}

\section{SUSYNTUPLA v2 User's Guide}
\label{sec-1}

\subsection{Setup}
\label{sec-1.1}

\subsubsection{Installation and Configuration}
\label{sec-1.1.1}

     Once you have extracted the source code to a folder of your choice, open
     the file setup.(c)sh in a text editor. Change the value of the variable,
     SUSY\_WORKING\_SW\_DIR to the absolute path of the folder containing the
     SUSYNTUPLA code. It should look something like this:

\begin{verbatim}
     setenv SUSY_WORKING_SW_DIR $HOME/susy_code

\end{verbatim}



     You may also alter the other variables to change the version of ROOT you
     are using or the Python path. If you are running on LXPLUS, these should
     already be correctly set. If you are running on your own machine, you may
     need to change the value of MY\_ROOTSYS.

     Finally, open the file config.inc. If you are running on LXPLUS, these
     values should be set correctly. If you are running on your own machine, you
     may need to change them. This will depend on how ROOT is installed on your
     machine. For Ubuntu, I have:

\begin{verbatim}
     ROOTSYS_INC=/usr/include/root
     ROOTSYS_LIB=/usr/lib
     ROOTSYS_BIN=/usr/bin
     PYTHON_INC=/usr/include/python2.6
     PYTHON_LIB=-lpython2.6
     BOOST_PYTHON_LIB=-lboost_python-mt

\end{verbatim}



     For LXPLUS, it should be:

\begin{verbatim}
     ROOTSYS_INC=$(ROOTSYS)/include
     ROOTSYS_LIB=$(ROOTSYS)/lib
     ROOTSYS_BIN=$(ROOTSYS)/bin
     PYTHON_INC=/usr/include/python2.3
     PYTHON_LIB=-lpython2.3
     BOOST_PYTHON_LIB=-lboost_python-mt

\end{verbatim}



     Now, everything should be configured correctly. Simply source the file
     setup.(c)sh and your environment will be setup.

\subsubsection{Compiling}
\label{sec-1.1.2}

    The code structure for V2 of the code has changed. Common aspects of the
    code are in the directory framework/. These are compiled into a shared
    library which libFrameworkSUSY.so which can then be linked against your own
    analysis code. All code specific to a given analysis should reside in a
    separate folder within the SUSY code structure. For instance the single
    lepton analysis would reside in the folder onelepton. Each analysis' code is
    built into a separate shared library which will be stored in the lib/ folder
    in the root of the codebase. Each module has two makefiles Debug and
    Release. Most of the time, the Debug file should be used so that the
    resulting library can be debugged.

    In order to build the onelepton module, you must first build the core
    framework by changing in to the framework directory and typing:


\begin{verbatim}
    make -f Debug

\end{verbatim}



    This compiles the core framework code. This should only be needed when the
    core code has been changed. NOTE: You do not need to make clean. To compile
    the onelepton code itself, change to the onelepton directory and type the
    same command.

\subsection{The Code}
\label{sec-1.2}

\subsubsection{The Python.cc file}
\label{sec-1.2.1}

    This file is used to expose the operations defined in your shared library so
    that they can be accessed from a Python script. The magic that connects C++
    to Python is done by the BOOST Python library so you should look at \href{http://www.boost.org/doc/libs/1_41_0/libs/python/doc/index.html}{the manual} for proper documentation.

    The essential element of this file is the class\_ directive. This has the
    following syntax:

\begin{verbatim}
    class_<ClassName,bases<BaseClassName>("PythonName",init<ArgType1,ArgType2...>());

\end{verbatim}



    This tells Python that the C++ class ClassName should be exposed as the
    Python class PythonName with a constructor that takes ArgType1,
    ArgType2\ldots{} as arguments. It also tells Python that the class ClassName
    inherits from BaseClassName. For the purposes of simply exposing an
    operation to Python, this should be enough. So to expose the NumComMuons
    operation we do the following:

\begin{verbatim}
    class_<Operation::NumComMuons, bases<Operation::_Base> >("OP_NumComMuons",
                                                           init<const std::string &,
                                                           UInt_t>());

\end{verbatim}



    This tells Python to register the class NumComMuons in the Operation
    namespace as Python class OP\_NumComMuons. The constructor will take a
    reference to a string and an unsinged integer. Also, the operation will inherit
    from the Operation::\_Base class.

    Once you have done this, simply rebuild the module and the class will be
    available from Python.


\subsection{Frequently Asked Questions}
\label{sec-1.3}

\subsubsection{I get weird Symbol Not Found errors}
\label{sec-1.3.1}

    These look a lot more confusing than they actually are. They occur because
    when the code is built as a shared library it will not complain about not
    being able to find bits of your code (since they may be in another shared
    library).

    To decipher them, use the c++filt command to untangle the problematic symbol
    name.

\begin{verbatim}
    c++filt <symbol_name>

\end{verbatim}


    This will tell you where the problem is in your code.

\end{document}