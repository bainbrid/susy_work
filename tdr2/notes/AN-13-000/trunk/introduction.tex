\clearpage
\section{Introduction \label{sec:intro}}

In this note we present an update of the search for a missing energy
signature in dijet and multi-jet events using the kinematic variable
\alphat, as first introduced in Refs.~\cite{Randall:2008rw,
  cms-pas-sus-08005, cms-pas-sus-09001} and described in
Sec.~\ref{sec:selection}. 

These results are based on a data sample of pp collisions collected in
2012 at a centre-of-mass energy of 8 TeV, which corresponds to an
integrated luminosity of 18.5$\pm$XX\fbinv. The data sample comprises
events recorded with the usual signal trigger for this analysis, plus
a ``parked'' trigger that allows the signal region phase space to
further enlarged with respect to the previous analysis.\footnote{The
  ``parked'' trigger was not available for the run period Run2012A,
  which is the reason why the intergrated luminosity is just short of
  the $\sim$19--20\fbinv as is typically the case for other analyses.}

A search for an excess of events in data over the Standard Model
expectation is performed in multi-jet final-states with significant
\met. The dominant background is multi-jet production, a manifestation
of quantum chromodynamics (QCD), which is suppressed by the \alphat
variable to a negligible level. To estimate the remaining significant
backgrounds, we make use of four data control samples: a \mj sample to
determine the background from \wj, \ttbar and single top events; a \gj
sample to determine the irreducible background from \znunu\ + jets
events; a \mmj sample that is also used to determine the \znunu\ +
jets background; and finally a multijet-enriched hadronic control
sample to determine any residual contribution from multijet
production.

\section{Parked data and compressed-spectrum models \label{sec:parked}}

The search focuses on event topologies in which new heavy particles
are pair-produced, each of which then decays to a weakly interacting
massive particle (WIMP) that remains undetected, thus leading to a
missing energy signature. In the case of SUSY, the candidate heavy
particles are squarks and gluinos and the WIMP candidate is the
lightest (and stable) neutralino $\chiznew_1$. 

Thus, this search requires at least two high-\pt jets and significant
\met in the final state. 

the search has been adapted to
 improve the sensitivity to final-state signatures rich in heavy
 quarks; 

The results presented below are interpreted
in the context of SUSY, although they are also applicable to other New
Physics scenarios that are characterised by a missing transverse
energy signature, such as Extra Dimensions and Little Higgs models. 

The results are interpreted using simplified model spectra
(SMS)~\cite{Alwall:2008ag,Alwall:2008va,sms}. A simplified model is
defined by an effective Lagrangian describing the interactions of a
small number of new particles, which can be equally well described by
a small number of observables, such as masses and cross-sections.
Simplified models are therefore particularly useful for evaluating
phase-space coverage of both individual searches and experiment-wide
search programs, as well as providing an excellent starting point for
characterizing positive signals of new physics.

\section{Changes with respect to HCP result\label{sec:changes}}

The analysis follows closely Ref.~\cite{RA1Paper2012}, which in turn
is based on Refs.~\cite{RA1Paper2011FULL, RA1Paper2011, RA1Paper} and
two simulation-based studies~\cite{cms-pas-sus-09001,
  cms-pas-sus-08005}. 

The main difference with respect to Ref.~\cite{RA1Paper2012}, as
mentioned above, is the extended phase space coverage of the signal
region through the addition of a ``parked'' trigger. Specifically, the
signal region is defined (in part) by a lower bound on the scalar sum
of the jet momenta observed in each even, which has been lowered from
275\gev to 200\gev. This change is described in more detail in
Secs.~\ref{sec:trigger} and \ref{sec:selection}.

In order to better control the level of background events, further
changes have been made. A new method, based on data control samples,
is now employed to determine any contribution from multijet
production. {\bf SITV? NJET? ALPHAT?}

To aid the reader, a full list of all changes with respect to the HCP
result is provided below.

\begin{itemize}
\item 
\end{itemize}

% \footnote{A further change has also been investigated and was
%   pre-approved by the SUSY group on 1$^{\textrm{st}}$ March 2012: the
%   hadronic signal region is enlarged to cover more signal phase space,
%   by extending the region to lower values of the \alphat
%   variable. This approach is explained further in
%   Appendix~\ref{app:2d-approach}, although it is not used by the
%   analysis presented in this note.}: the search has been adapted to
% improve the sensitivity to final-state signatures rich in heavy
% quarks; and a new data control sample for background estimation is
% added. 

