%%%%%%%%%%%%%%%%%%%%%%%%%%%%%%%%%%%%%%%%%%%%%%%%%%%%%%%%%%%%%%%%%%%%%%%%%%%%%%%%
%%%%%%%%%%%%%%%%%%%%%%%%%%%%%%%%%%%%%%%%%%%%%%%%%%%%%%%%%%%%%%%%%%%%%%%%%%%%%%%%
%%%%%%%%%%%%%%%%%%%%%%%%%%%%%%%%%%%%%%%%%%%%%%%%%%%%%%%%%%%%%%%%%%%%%%%%%%%%%%%%

\clearpage
\subsection{AN-12-356 (Early 2012 analysis for HCP based on 12\fbinv @ 8 TeV)}

Negligible contamination from QCD multi-jet events in the signal
region is expected due to the application of the cuts $\scalht >
275\gev$ and $\alphat > 0.55$. Any residual leakage is removed using
various cleaning filters, as described in
Sec.~\ref{sec:had-signal}. These filters include the requirement
$\mht/\met < 1.25$, which ensures that jets below the $\Et$ threshold
do not contribute significantly to \mht.

Although the signal region is expected to be free from multi-jet
events, a conservative approach is taken and the likelihood model is
given the freedom to estimate any potential contamination from
multi-jet events. This is achieved by considering the
\scalht-dependence of the variable \RaT, which is defined as the ratio
of events above and below the threshold value of $\alphat^{\rm
  cut}=0.55$ for a given \scalht bin. This dependence is modelled
accurately by a (falling) exponential function of \scalht, which takes
the form $\RaT(\scalht) = A\mathrm{e}^{-k \, H_\text{T}}$, where the
parameters $A$ and $k$ ($\gev^{-1}$) are normalisation and exponential
decay constants, respectively. Independent decay constant ($k$) and
normalisation ($A_{n_b}$) parameters are provided for each b-tag
multiplicity category. The same exponential model has been used in
2010 and 2011~\cite{RA1Paper2011,RA1Paper,RA1PAS2011,RA1PAS2012} and
has been validated using both MC simulation and data. Further
data-driven studies are detailed below. The formal description of the
likelihood model can be found in Sec.~\ref{sec:statistics}. The
exponential behaviour is a result of several features, namely: the
improving jet energy resolutions with increasing \scalht; the reducing
impact of pathological effects at increasing energy scales; and, for
the region $\scalht > 375\gev$, the average jet multiplicity per bin
increases slowly with \scalht, which results in a narrower \alphat
distribution (peaking at 0.5) due to the increased combinatorics in
the $\Delta\scalht$ jet recombination scheme.

Due to the signal region definition and this exponentially falling
behaviour, MC studies demonstrate that the expected contamination from
multi-jet events falls quickly from a negligible contribution to zero
in the region $400-500\gev$. Therefore, the accuracy of the
exponential modelling at high \scalht (\ie, $\scalht > 575\gev$) is
not of paramount importance, although the model has been proven valid
for the full \scalht range used in the analysis, as demonstrated in
data below.

Maximum likelihood (ML) values of the parameters $k$ and $A$ are
determined by the likelihood fit. However, the value of the decay
constant $k$ is constrained first via measurements in
multi-jet$-$enriched data side-bands. This is done in order to account
correctly for the background composition, namely QCD multi-jets or
backgrounds with genuine \met. Cross-checks demonstrate that removing
this constraint altogether has little effect on the total SM
background expectations determined by the likelihood fit. The data
samples in the side-band regions are collected using the set of
prescaled \httrigger triggers described in Sec.~\ref{sec:triggers}.

\begin{figure}[!h]
  \begin{center}
    \includegraphics[width=0.5\textwidth,angle=0]{figures/qcd/AN-12-356/side-bands.pdf}
    \caption{QCD side-bands.}
    \label{fig:side-bands}
  \end{center}
\end{figure}

Figure~\ref{fig:side-bands} depicts a simple cartoon that defines the
data side-bands used to constrain the value of the parameter $k$. The
side-bands are defined using the variables \alphat and
$\mht/\met$. The signal region is labelled in the cartoon and is
defined by $\alphat > 0.55$ and $\mht/\met < 1.25$. Region $B$ is a
side-band defined by inverting the \alphat cut (\ie $\alphat <
0.55$). Region $C$ is defined by inverting also the $\mht/\met$ cut
(\ie $\mht/\met > 1.25$) and splitting into three slices in \alphat:
$0.52 < \alphat < 0.53$, $0.53 < \alphat < 0.54$, and $0.54 < \alphat
< 0.55$ (labelled as $C_1$, $C_2$, and $C_3$, respectively). By
inverting one and then the other cut on these two variables, the
sample is increasingly enriched in QCD multi-jet events (loosely in
the direction of the arrow). For the purposes of this preliminary
result, the constraint on $k$ is taken from the 2011 analysis, which
is determined to be $[ -2.96 \pm 0.61 (stat.) \pm 0.46 (syst.)  ]
\times 10^{-2}\gev^{-1}$~\cite{RA1PAS2011}.

%%%%%%%%%%%%%%%%%%%%%%%%%%%%%%%%%%%%%%%%%%%%%%%%%%%%%%%%%%%%%%%%%%%%%%%%%%%%%%%%
%%%%%%%%%%%%%%%%%%%%%%%%%%%%%%%%%%%%%%%%%%%%%%%%%%%%%%%%%%%%%%%%%%%%%%%%%%%%%%%%
%%%%%%%%%%%%%%%%%%%%%%%%%%%%%%%%%%%%%%%%%%%%%%%%%%%%%%%%%%%%%%%%%%%%%%%%%%%%%%%%

\clearpage
\subsection{AN-12-052 (HPA analysis for ICHEP 2012 based on 4\fbinv @ 8 TeV)}

Negligible contamination from QCD multi-jet events in the signal
region is expected due to the application of the cuts $\scalht >
275\gev$ and $\alphat > 0.55$. Any residual leakage is removed using
various cleaning filters, as described in
Sec.~\ref{sec:had-signal}. These filters include the requirement
$\mht/\met < 1.25$, which ensures that jets below the $\Et$ threshold
do not contribute significantly to \mht.

Although the signal region is expected to be free from multi-jet
events, a conservative approach is taken and the likelihood model is
given the freedom to estimate any potential contamination from
multi-jet events. This is achieved by considering the
\scalht-dependence of the variable \RaT, which is defined as the ratio
of events above and below the threshold value of $\alphat^{\rm
  cut}=0.55$ for a given \scalht bin. This dependence is modelled
accurately by a (falling) exponential function of \scalht, which takes
the form $\RaT(\scalht) = A\mathrm{e}^{-k \, H_\text{T}}$, where the
parameters $A$ and $k$ ($\gev^{-1}$) are normalisation and exponential
decay constants, respectively. Independent decay constant ($k$) and
normalisation ($A_{n_b}$) parameters are provided for each b-tag
multiplicity category. The same exponential model has been used in
2010 and 2011~\cite{RA1Paper2011,RA1Paper,RA1PAS2011} and has been
validated using both MC simulation and data. Further data-driven
studies are detailed below. The formal description of the likelihood
model can be found in Sec.~\ref{sec:statistics}. The exponential
behaviour is a result of several features, namely: the improving jet
energy resolutions with increasing \scalht; the reducing impact of
pathological effects at increasing energy scales; and, for the region
$\scalht > 375\gev$, the average jet multiplicity per bin increases
slowly with \scalht, which results in a narrower \alphat distribution
(peaking at 0.5) due to the increased combinatorics in the
$\Delta\scalht$ jet recombination scheme.

Due to the signal region definition and this exponentially falling
behaviour, MC studies demonstrate that the expected contamination from
multi-jet events falls quickly from a negligible contribution to zero
in the region $400-500\gev$. Therefore, the accuracy of the
exponential modelling at high \scalht (\ie, $\scalht > 575\gev$) is
not of paramount importance, although the model has been proven valid
for the full \scalht range used in the analysis, as demonstrated in
data below.

Maximum likelihood (ML) values of the parameters $k$ and $A$ are
determined by the likelihood fit. However, the value of the decay
constant $k$ is constrained first via measurements in
multi-jet$-$enriched data side-bands. This is done in order to account
correctly for the background composition, namely QCD multi-jets or
backgrounds with genuine \met. Cross-checks demonstrate that removing
this constraint altogether has little effect on the total SM
background expectations determined by the likelihood fit. The data
samples in the side-band regions are collected using the set of
prescaled \httrigger triggers described in Sec.~\ref{sec:triggers}.

\begin{figure}[!h]
  \begin{center}
    \includegraphics[width=0.5\textwidth,angle=0]{figures/qcd/AN-12-052/side-bands.pdf}
    \caption{QCD side-bands.}
    \label{fig:side-bands}
  \end{center}
\end{figure}

Figure~\ref{fig:side-bands} depicts a simple cartoon that defines the
data side-bands used to constrain the value of the parameter $k$. The
side-bands are defined using the variables \alphat and
$\mht/\met$. The signal region is labelled in the cartoon and is
defined by $\alphat > 0.55$ and $\mht/\met < 1.25$. Region $B$ is a
side-band defined by inverting the \alphat cut (\ie $\alphat <
0.55$). Region $C$ is defined by inverting also the $\mht/\met$ cut
(\ie $\mht/\met > 1.25$) and splitting into three slices in \alphat:
$0.52 < \alphat < 0.53$, $0.53 < \alphat < 0.54$, and $0.54 < \alphat
< 0.55$ (labelled as $C_1$, $C_2$, and $C_3$, respectively). By
inverting one and then the other cut on these two variables, the
sample is increasingly enriched in QCD multi-jet events (loosely in
the direction of the arrow). For the purposes of this preliminary
result, the constraint on $k$ is taken from the 2011 analysis, which
is determined to be $[ -2.96 \pm 0.61 (stat.) \pm 0.46 (syst.)  ]
\times 10^{-2}\gev^{-1}$~\cite{RA1PAS2011}.

% A further multi-jet$-$enriched side-band, labelled $D$, is defined by
% inverting just the $\mht/\met$ cut; this region is not used to
% constrain the value of the parameter $k$, but instead is used in a
% further cross-check on the validity of the exponential model,
% described below.

% Figures~\ref{fig:side-band-fits-lin} shows the resulting behaviour of
% \RaT as a function of \scalht for the side-band regions $B$, $C_1$,
% $C_2$, and $C_3$. The large uncertainties on the measurements are due
% to the large prescales applied to the \httrigger
% triggers. Measurements are made in the region $275 < \scalht < 575$
% only.\footnote{A possible extension to this analysis is to extend the
%   signal region to include lower values of \alphat, namely covering
%   the regions $0.53 < \alphat < 0.55$ for $\scalht > 575\gev$ and
%   $0.52 < \alphat < 0.53$ for $\scalht > 775\gev$.%, as described in Appendix~\ref{app:2d-approach}. 
%   This limits the data side-bands to the \scalht region defined in the
%   text.} An exponential fit to the data is made for each side-band
% region. Table~\ref{tab:expo-qcd-fits} summarises the best fit
% parameter values for $k$ and the associated $p$-values.

% \begin{table}[!h]
%   \caption{\label{tab:expo-qcd-fits} Best fit values for the
%     parameters $k$ as obtained from the regions $B$, $C_1$, $C_2$, and
%     $C_3$. The latter three measurements are used to calculate a
%     weighted mean (identified as region $C$). Also quoted is the
%     maximum likelihood value of the parameter $k$ given by the
%     simultaneous fit using the sample defined by region $D$. Quoted
%     errors are statistical only.}
%   \centering
%   \footnotesize
%   \begin{tabular}{ ccc }
%     \hline
%     Side-band region     & $k (\times 10^{-2}\gev^{-1})$ & $p$-value \\ [0.5ex]
%     \hline                            
%     $B$                  & $0.72\pm0.47$                 & 0.20      \\ 
%     $C_1$                & $0.80\pm0.43$                 & 0.36      \\ 
%     $C_2$                & $1.29\pm0.20$                 & 0.71      \\ 
%     $C_3$                & $0.89\pm0.10$                 & 0.88      \\ 
%     \hline                            
%     $C$ (weighted mean)  & $0.96\pm0.09$                 & -         \\ 
%     %$D$ (likelihood fit) & $1.31\pm0.09$                 & -         \\
%     \hline
%   \end{tabular}
% \end{table}

% The best fit value for the parameter $k$ as obtained from region $B$
% is $(0.72 \pm 0.47) \times 10^{-2}\gev^{-1}$, which is taken as the
% central value for the constraint to be used in the fit. The assumption
% that this approach gives an unbiased estimator for $k$ is motivated by
% the fact that the event kinematics in the region $0.52 < \alphat <
% 0.55$ and $\alphat > 0.55$ are similar. This is validated below.

% The best fit values for the parameters $k$ as obtained from the
% regions $C_1$, $C_2$, and $C_3$ are used to estimate a systematic
% uncertainty on the central value. The best fit values exhibit no
% strong dependence on the \alphat slice, supporting the assumption
% above that region $B$ provides an unbiased estimator of $k$. However,
% the observed (albeit not significant) variations between the different
% \alphat slices in region $C$ are used to determine a systematic
% uncertainty for $k$. The weighted mean and standard deviation of these
% three measurements is calculated to be $(1.31\pm0.26) \times
% 10^{-2}\gev^{-1}$. The relative error on this measurement is 20\%,
% which is applied to the central value to give an estimate of the
% systematic uncertainty.

% In summary, the aforementioned data side-bands are used to provide a
% constrained value of $k$ as input to the likelihood fit, the value of
% which is determined to be $[ 0.72 \pm 0.47 (stat.) \pm 0.07 (syst.)
% ] \times 10^{-2}\gev^{-1}$. These uncertainties are used as penalty
% terms in the likelihood model, as described in
% Sec.~\ref{sec:statistics}.

% \begin{figure}[!h]
%   \begin{center}
%     \includegraphics[width=0.5\textwidth,angle=0]{figures/qcd_plots/qcd-fit.pdf}
%     \caption{Comparison of the observed yields and SM expectations
%       given by the simultaneous fit in bins of \scalht for the
%       side-band region $D$. No requirement on the number of b-jets is
%       made. Shown are the observed event yields in data (black dots
%       with error bars representing the statistical uncertainties) and
%       the expectations given by the simultaneous fit for the
%       \znunu+jets process (orange dotted-dashed line); the sum of all
%       processes with genuine \met, which are primarily \ttbar, W+jets,
%       and \znunu+jets (dark blue long-dashed line); and the sum of QCD
%       and all aforementioned SM processes (light blue solid line).}
%     \label{fig:qcd-fit}
%   \end{center}
% \end{figure}

% One further cross-check is performed using the side-band $D$. In this
% case, the likelihood fit is performed for this multi-jet$-$enriched
% region, and no constraint is applied to $k$, which is allowed to be
% determined freely by the fit. The fit is performed over full \scalht
% region, \ie it is not limited to $275 < \scalht < 575\gev$, and no
% requirement on the number of reconstructed b-jets per event is
% made. Figure~\ref{fig:qcd-fit} shows the resulting fit, which gives a
% ML value of $(1.31\pm0.09) \times 10^{-2}\gev^{-1}$.
% % and a $p$-value of 0.57. 
% This final cross-check provides further supporting evidence for two
% crucial aspects of the method described above. First, the fit provides
% evidence that the exponential function used in the likelihood model is
% valid for the entire \scalht regime.
% %, as demonstrated by the $p$-value. 
% Second, the ML value for $k$ is in excellent agreement with the
% weighted mean obtained from the region $C$ (the two values can be
% compared directly in Table~\ref{tab:expo-qcd-fits}). This further
% supports the assumption that region $B$ (with $0.52 < \alphat < 0.55$)
% can provide an unbiased estimator for $k$ in the signal region
% ($\alphat > 0.55$).

%%%%%%%%%%%%%%%%%%%%%%%%%%%%%%%%%%%%%%%%%%%%%%%%%%%%%%%%%%%%%%%%%%%%%%%%%%%%%%%%
%%%%%%%%%%%%%%%%%%%%%%%%%%%%%%%%%%%%%%%%%%%%%%%%%%%%%%%%%%%%%%%%%%%%%%%%%%%%%%%%
%%%%%%%%%%%%%%%%%%%%%%%%%%%%%%%%%%%%%%%%%%%%%%%%%%%%%%%%%%%%%%%%%%%%%%%%%%%%%%%%

\clearpage
\subsection{AN-11-517 (2011 analysis of 5\fbinv @ 7 TeV)}

Negligible contamination from QCD multi-jet events in the signal
region is expected due to the application of the cuts $\scalht >
275\gev$ and $\alphat > 0.55$. Any residual leakage is removed using
various cleaning filters, as described in
Sec.~\ref{sec:had-signal}. These filters include the requirement
$\mht/\met < 1.25$, which ensures that jets below the $\Et$ threshold
do not contribute significantly to \mht.

Although the signal region is expected to be free from multi-jet
events, a conservative approach is taken and the likelihood model is
given the freedom to estimate any potential contamination from
multi-jet events. This is achieved by considering the
\scalht-dependence of the variable \RaT, which is defined as the ratio
of events above and below the threshold value of $\alphat^{\rm
  cut}=0.55$ for a given \scalht bin. This dependence is modelled
accurately by a (falling) exponential function of \scalht, which takes
the form $\RaT(\scalht) = A_{n_b}\mathrm{e}^{-k \, H_\text{T}}$, where
the parameters $A_{n_b}$ and $k$ ($\gev^{-1}$) are normalisation and
exponential decay constants, respectively. A unique decay constant
($k$) is assumed for all b-tag multiplicity categories and an
independent normalisation parameter ($A_{n_b}$) is provided for each
b-tag multiplicity. The same exponential model has been used in 2010
and 2011~\cite{RA1Paper2011,RA1Paper} and has been validated using
both MC simulation and data. Further data-driven studies are detailed
below. The formal description of the likelihood model can be found in
Sec.~\ref{sec:statistics}.

The exponential behaviour is a result of several features, namely: the
improving jet energy resolutions with increasing \scalht; the reducing
impact of pathological effects at increasing energy scales; and, for
the region $\scalht > 375\gev$, the average jet multiplicity per bin
increases slowly with \scalht, which results in a narrower \alphat
distribution (peaking at 0.5) due to the increased combinatorics in
the $\Delta\scalht$ jet recombination scheme. 

Due to the signal region definition and this exponentially falling
behaviour, MC studies demonstrate that the expected contamination from
multi-jet events falls quickly from a negligible contribution to zero
in the region $400-500\gev$. Therefore, the accuracy of the
exponential modelling at high \scalht (\ie, $\scalht > 575\gev$) is
not of paramount importance, although the model has been proven valid
for the full \scalht range used in the analysis, as demonstrated in
data below.

Maximum likelihood (ML) values of the parameters $k$ and $A_{n_b}$ are
determined by the likelihood fit. However, the value of decay constant
$k$ is constrained first via measurements in multi-jet$-$enriched data
side-bands. This is done in order to account correctly for the
background composition, namely QCD multi-jets or backgrounds with
genuine \met. Cross-checks demonstrate that removing this constraint
altogether has little effect on the total SM background expectations
determined by the likelihood fit. The data samples in the side-band
regions are collected using the set of prescaled \httrigger triggers
described in Sec.~\ref{sec:triggers}.

\begin{figure}[!h]
  \begin{center}
    \includegraphics[width=0.5\textwidth,angle=0]{figures/qcd/AN-11-517/side-bands.pdf}
    \caption{QCD side-bands.}
    \label{fig:side-bands}
  \end{center}
\end{figure}

Figure~\ref{fig:side-bands} depicts a simple cartoon that defines the
data side-bands used to constrain the value of the parameter $k$. The
side-bands are defined using the variables \alphat and
$\mht/\met$. The signal region is labelled in the cartoon and is
defined by $\alphat > 0.55$ and $\mht/\met < 1.25$. Region $B$ is a
side-band defined by inverting the \alphat cut (\ie $\alphat <
0.55$). Region $C$ is defined by inverting also the $\mht/\met$ cut
(\ie $\mht/\met > 1.25$) and splitting into three slices in \alphat:
$0.52 < \alphat < 0.53$, $0.53 < \alphat < 0.54$, and $0.54 < \alphat
< 0.55$ (labelled as $C_1$, $C_2$, and $C_3$, respectively). By
inverting one and then the other cut on these two variables, the
sample is increasingly enriched in QCD multi-jet events (loosely in
the direction of the arrow). A further multi-jet$-$enriched side-band,
labelled $D$, is defined by inverting just the $\mht/\met$ cut; this
region is not used to constrain the value of the parameter $k$, but
instead is used in a further cross-check on the validity of the
exponential model, described below.

Figures~\ref{fig:side-band-fits-lin} shows the resulting behaviour of
\RaT as a function of \scalht for the side-band regions $B$, $C_1$,
$C_2$, and $C_3$. The large uncertainties on the measurements are due
to the large prescales applied to the \httrigger
triggers. Measurements are made in the region $275 < \scalht < 575$
only.\footnote{A possible extension to this analysis is to extend the
  signal region to include lower values of \alphat, namely covering
  the regions $0.53 < \alphat < 0.55$ for $\scalht > 575\gev$ and
  $0.52 < \alphat < 0.53$ for $\scalht > 775\gev$, as described in
  Appendix~\ref{app:2d-approach}. This limits the data side-bands to
  the \scalht region defined in the text.} An exponential fit to the
data is made for each side-band region. Table~\ref{tab:expo-qcd-fits}
summarises the best fit parameter values for $k$ and the associated
$p$-values.

\begin{table}[!h]
  \caption{\label{tab:expo-qcd-fits} Best fit values for the
    parameters $k$ as obtained from the regions $B$, $C_1$, $C_2$, and
    $C_3$. The latter three measurements are used to calculate a
    weighted mean (identified as region $C$). Also quoted is the
    maximum likelihood value of the parameter $k$ given by the
    simultaneous fit using the sample defined by region $D$. Quoted
    errors are statistical only.}
  \centering
  \footnotesize
  \begin{tabular}{ ccc }
    \hline
    Side-band region     & $k (\times 10^{-2}\gev^{-1})$ & $p$-value \\ [0.5ex]
    \hline                            
    $B$                  & $2.96\pm0.64$                 & 0.24      \\ 
    $C_1$                & $1.19\pm0.45$                 & 0.93      \\ 
    $C_2$                & $1.47\pm0.37$                 & 0.42      \\ 
    $C_3$                & $1.17\pm0.55$                 & 0.98      \\ 
    \hline                            
    $C$ (weighted mean)  & $1.31\pm0.26$                 & -         \\ 
    $D$ (likelihood fit) & $1.31\pm0.09$                 & 0.57      \\
    \hline
  \end{tabular}
\end{table}

The best fit value for the parameter $k$ as obtained from region $B$
is $(2.96\pm0.64) \times 10^{-2}\gev^{-1}$, which is taken as the
central value for the constraint to be used in the fit. The assumption
that this approach gives an unbiased estimator for $k$ is motivated by
the fact that the event kinematics in the region $0.52 < \alphat <
0.55$ and $\alphat > 0.55$ are similar. This is validated below.

The best fit values for the parameters $k$ as obtained from the
regions $C_1$, $C_2$, and $C_3$ are used to estimate a systematic
uncertainty on the central value. The best fit values exhibit no
strong dependence on the \alphat slice, supporting the assumption
above that region $B$ provides an unbiased estimator of $k$. However,
the observed (albeit not significant) variations between the different
\alphat slices in region $C$ are used to determine a systematic
uncertainty for $k$. The weighted mean and standard deviation of these
three measurements is calculated to be $(1.31\pm0.26) \times
10^{-2}\gev^{-1}$. The relative error on this measurement is 20\%,
which is applied to the central value to give an estimate of the
systematic uncertainty.

In summary, the aforementioned data side-bands are used to provide a
constrained value of $k$ as input to the likelihood fit, the value of
which is determined to be $[ -2.96 \pm 0.61 (stat.) \pm 0.46 (syst.)
] \times 10^{-2}\gev^{-1}$. These uncertainties are used as penalty
terms in the likelihood model, as described in
Sec.~\ref{sec:statistics}.

\begin{figure}[!h]
  \begin{center}
    \includegraphics[width=0.5\textwidth,angle=0]{figures/qcd/AN-11-517/qcd-fit.pdf}
    \caption{Comparison of the observed yields and SM expectations
      given by the simultaneous fit in bins of \scalht for the
      side-band region $D$. No requirement on the number of b jets is
      made. Shown are the observed event yields in data (black dots
      with error bars representing the statistical uncertainties) and
      the expectations given by the simultaneous fit for the
      \znunu+jets process (orange dotted-dashed line); the sum of all
      processes with genuine \met, which are primarily \ttbar, W+jets,
      and \znunu+jets (dark blue long-dashed line); and the sum of QCD
      and all aforementioned SM processes (light blue solid line).}
    \label{fig:qcd-fit}
  \end{center}
\end{figure}

One further cross-check is performed using the side-band $D$. In this
case, the likelihood fit is performed for this multi-jet$-$enriched
region, and no constraint is applied to $k$, which is allowed to be
determined freely by the fit. The fit is performed over full \scalht
region, \ie it is not limited to $275 < \scalht < 575\gev$, and no
requirement on the number of reconstructed b jets per event is
made. Figure~\ref{fig:qcd-fit} shows the resulting fit, which gives a
ML value of $(1.31\pm0.09) \times 10^{-2}\gev^{-1}$.
% and a $p$-value of 0.57. 
This final cross-check provides further supporting evidence for two
crucial aspects of the method described above. First, the fit provides
evidence that the exponential function used in the likelihood model is
valid for the entire \scalht regime.
%, as demonstrated by the $p$-value. 
Second, the ML value for $k$ is in excellent agreement with the
weighted mean obtained from the region $C$ (the two values can be
compared directly in Table~\ref{tab:expo-qcd-fits}). This further
supports the assumption that region $B$ (with $0.52 < \alphat < 0.55$)
can provide an unbiased estimator for $k$ in the signal region
($\alphat > 0.55$).

\clearpage
\subsubsection{Additional  information on QCD multi-jet background estimation} 
\label{app:qcd}

% \begin{figure}[!h]
%   \begin{center}
%     \subfigure[Side-band region $B$.]{
%       \label{fig:side-band-b}
%       \includegraphics[width=0.48\textwidth]{figures/qcd_plots/RaT052log.pdf} 
%     } 
%     \subfigure[Side-band region $C_1$.]{
%       \label{fig:side-band-c1}
%       \includegraphics[width=0.48\textwidth]{figures/qcd_plots/RaT54log.pdf} 
%     } \\
%     \subfigure[Side-band region $C_2$.]{
%       \label{fig:side-band-c2}
%       \includegraphics[width=0.48\textwidth]{figures/qcd_plots/RaT53log.pdf} 
%     } 
%     \subfigure[Side-band region $C_3$.]{
%       \label{fig:side-band-c3}
%       \includegraphics[width=0.48\textwidth]{figures/qcd_plots/RaT52log.pdf} 
%     } \\
%     \caption{\label{fig:side-band-fits}$\RaT(\scalht)$ and exponential fit
%       for various data side-bands. Logarithmic y-axis scale.}
%   \end{center}
% \end{figure}

% \newpage
\begin{figure}[!h]
  \begin{center}
    \subfigure[Side-band region $B$.]{
      \label{fig:side-band-b-lin}
      \includegraphics[width=0.48\textwidth]{figures/qcd/AN-11-517/RaT052lin.pdf} 
    } 
    \subfigure[Side-band region $C_1$.]{
      \label{fig:side-band-c1-lin}
      \includegraphics[width=0.48\textwidth]{figures/qcd/AN-11-517/RaT54lin.pdf} 
    } \\
    \subfigure[Side-band region $C_2$.]{
      \label{fig:side-band-c2-lin}
      \includegraphics[width=0.48\textwidth]{figures/qcd/AN-11-517/RaT53lin.pdf} 
    } 
    \subfigure[Side-band region $C_3$.]{
      \label{fig:side-band-c3-lin}
      \includegraphics[width=0.48\textwidth]{figures/qcd/AN-11-517/RaT52lin.pdf} 
    } \\
    \caption{\label{fig:side-band-fits-lin}$\RaT(\scalht)$ and exponential fit
      for various data side-bands. Linear y-axis scale.}
  \end{center}
\end{figure}

%%%%%%%%%%%%%%%%%%%%%%%%%%%%%%%%%%%%%%%%%%%%%%%%%%%%%%%%%%%%%%%%%%%%%%%%%%%%%%%%
%%%%%%%%%%%%%%%%%%%%%%%%%%%%%%%%%%%%%%%%%%%%%%%%%%%%%%%%%%%%%%%%%%%%%%%%%%%%%%%%
%%%%%%%%%%%%%%%%%%%%%%%%%%%%%%%%%%%%%%%%%%%%%%%%%%%%%%%%%%%%%%%%%%%%%%%%%%%%%%%%

\clearpage
\subsection{AN-11-244 (Early 2011 analysis of 1.1\fbinv @ 7 TeV)}

\subsubsection{\scalht Dependence of \RaT  \label{sec:ht-scaling}}

The ratio $\RaT = N^{\alt > \theta}/N^{\alt < \theta}$ exhibits no
dependence on \scalht if $\theta$ is chosen such that the numerator of
the ratio in all \scalht bins is dominated by \ttbar, W +jets and
\znunu +jets events (referred to in the following as EWK) and there is
no significant contribution from events from QCD multi-jet
production~\cite{RA1Paper}. This is demonstrated in
Figure~\ref{fig:rat_vs_ht}, using MC simulations for the cut value
$\theta = 0.55$ over the range $275 < \scalht < 975 \gev$.

\begin{figure}[!h]
  \begin{center}
    \includegraphics[width = 0.48\textwidth]{figures/qcd/AN-11-244/Ratio_Multi2Incl_AlphaT55.pdf}
    \includegraphics[width = 0.48\textwidth]{figures/qcd/AN-11-244/Syst_Multi2Incl_AlphaT55.pdf}
    \caption{\label{fig:rat_vs_ht} (Left) The dependence of \RaT on
      \HT for events with N$_{\mathrm{jet}} \geq 2$. (Right) Dependence of \RaT on
      \HT when varying the effective cross-section of the four major
      EWK background components individually by $\pm$15\%. (Markers
      are artificially offset for clarity.)  }
  \end{center}
\end{figure}

One important ingredient in the \RaT method is the scaling of the jet
\pt thresholds in the low \scalht bins to maintain jet multiplicities
and thus comparable event kinematics and topologies in the different
\scalht bins. This is especially important in the case of the \ttbar
background, which have on average a higher jet multiplicity then the
other EWK backgrounds.  If the jet \pt thresholds were not scaled
relative to \HT in the lowest \HT bins, \ttbar events would exhibit a
turn-on behaviour at low \HT before falling off exponentially like the
other SM backgrounds at high \HT.  Thus, in accordance with the 2010
analysis, the jet \pt thresholds are scaled only for the first two
\scalht bins, and remain fixed for all subsequent bins. The thresholds
are listed in Table~\ref{tab:results-HT}.  

Figure~\ref{fig:rat_vs_ht} (left) shows the dependence of \RaT on
\scalht, as measured in data and also obtained from MC simulations of
SM, and SM plus the SUSY benchmark points LM4 or LM6. The data (SM
expectations) are consistent with the flat hypothesis, with p-values
of 0.29 (0.50).  The SM+LM4 and SM+LM6 are not consistent with
constant $\RaT(\scalht)$, thus demonstrating that the presences of a
SUSY signal would result in a significant deviation from the
hypothesis of RaT being flat with \scalht .  As will be discussed in
more detail in Sec.~\ref{sec:statistics}, we also pursue an
alternative approach which allows for a small QCD contribution in the
low \HT regions. However, we have found no evidence in the 2011 data
that would invalidate the QCD free hypothesis, which in turn is
assumed to lead to \RaT being constant with \scalht .

Figure~\ref{fig:rat_vs_ht} (right) demonstrates the independence of
\RaT on \scalht, based on MC simulations, even when varying the
effective cross-section of the four major EWK background components
individually by as much as $\pm$15\%, which reflects our current
knowledge of the cross sections for these
backgrounds~\cite{top-xs,w-xs}. In each case, the behaviour is always
consistent with the flat hypothesis, with a p-value of at least
0.47. Studies with larger variations of $\pm$50\% also lead to
p-values that are consistent with the flat hypothesis. This is how the
assumption of flat behaviour is tested against the systematic
uncertainties associated with the cross-section measurements of the
different EWK backgrounds.

In 2010, a cut-based approach was used, in which an extrapolation from
a low-\scalht control region ($250 \GeV < \scalht < 350 \GeV$) into
the \scalht signal region ($\scalht > 350 \GeV$) was performed in
order to estimate the SM background. In the current analysis of the
2011 data, a shape analysis over the entire \HT $>$ 275 \GeV region is
carried out.

%%%%%%%%%%%%%%%%%%%%%%%%%%%%%%%%%%%%%%%%%%%%%%%%%%%%%%%%%%%%%%%%%%%%%%%%%%%%%%%%
%%%%%%%%%%%%%%%%%%%%%%%%%%%%%%%%%%%%%%%%%%%%%%%%%%%%%%%%%%%%%%%%%%%%%%%%%%%%%%%%
%%%%%%%%%%%%%%%%%%%%%%%%%%%%%%%%%%%%%%%%%%%%%%%%%%%%%%%%%%%%%%%%%%%%%%%%%%%%%%%%

\clearpage
\subsection{AN-10-242 (2010 analysis of 35\pbinv @ 7 TeV)}}

\subsubsection{Total background prediction using lower $H_{\mathrm{T}}$ control regions \label{sec:lowHTcontrol}}

\newcommand{\ptone}{\ensuremath{P_{\mathrm{T1}}}\xspace}
\newcommand{\pttwo}{\ensuremath{P_{\mathrm{T2}}}\xspace}
\newcommand{\ptall}{\ensuremath{P_{\mathrm{T}}}\xspace}
\newcommand{\aT}{\ensuremath{\alpha_{\mathrm{T}}}\xspace}
\newcommand{\raT}{\ensuremath{R_{\aT}}}
\newcommand{\RaT}{\ensuremath{R_{\aT}}\xspace}
%\newcommand{\pt}{\ensuremath{p_{\mathrm{T}}}\xspace}
\newcommand{\Npre}{\ensuremath{N_{\mathrm{predicted}}}\xspace}
\newcommand{\Nobs}{\ensuremath{N_{\mathrm{observed}}}\xspace}
\newcommand{\mymet}{\ensuremath{\eslash_{\mathrm{T}}}\xspace}

The method presented here is based on one documented in
\cite{cms-pas-sus-09001}, known as the ``$\eta$-$H_{\mathrm{T}}$''
method, that was designed primarily to establish the presence of a
signal incompatible with SM expectation, without breaking the SM
backgrounds down into their individual components.

The original method relies on the expectation that SUSY signal events
are produced more centrally in pseudo-rapidity with respect to QCD,
for which the main production mechanism is $t$-channel exchange, and
other SM backgrounds. The pseudo-rapidity of the leading jet, $\eta$,
is used as a measure of the centrality of an event. An excess is
identified through the dependence on $|\eta|$ of the ratio of events
passing a cut of $\alpha_{\mathrm{T}} > 0.55$. This dependence is
enhanced with increasing $H_{\mathrm{T}}$ in the presence of signal,
as shown in Fig.~\ref{fig:eta_ht_signal}. No dependence is observed
for SM backgrounds only, as indicated in Fig.~\ref{fig:eta_ht_bkgds}.

\begin{figure}[!h]
  \begin{center} 
    \subfigure[\label{fig:eta_ht_signal}]{\includegraphics[angle=90,width=0.45\linewidth]{figures/qcd/AN-10-242/SUS-09-001-Fig7a.pdf} }
    \subfigure[\label{fig:eta_ht_bkgds}]{\includegraphics[angle=90,width=0.45\linewidth]{figures/qcd/AN-10-242/SUS-09-001-Fig7b.pdf} }
    \caption{\label{fig:eta_ht} The ratio of events passing a cut of
      $\alpha_{\mathrm{T}} > 0.55$ plotted as a function of $|\eta|$
      of the leading jet for three different $H_{\mathrm{T}}$ bins, in
      the presence of an LM1 signal (a) and for SM backgrounds only
      (b), for an integrated luminosity of 100~$\mathrm{pb^{-1}}$ at
      $\sqrt{s} = 10~\mathrm{TeV}$. Taken from
      \cite{cms-pas-sus-09001}.}
  \end{center}
\end{figure}

Unfortunately, the size of the 2010 dataset is insufficient for a
categorisation in bins of $|\eta|$. Therefore, we have investigated
modifications to this procedure that are more suitable for a smaller
dataset. More specifically, we concentrate on the behaviour of the
ratio of events passing a cut on $\alpha_{\mathrm{T}}$ as a function
of $H_{\mathrm{T}}$ and do not consider the dependence on $\eta$.

\subsubsection{Methodology and motivation\label{sec:ht_scaling_method}}

An inclusive prediction of the SM background in the signal region of
$H_{\mathrm{T}} > 350~\mathrm{GeV}$ is derived directly from data by
making use of two control regions of lower $H_{\mathrm{T}}$:
$250~\mathrm{GeV} < H_{\mathrm{T}} \leq 300~\mathrm{GeV}$ and
$300~\mathrm{GeV} < H_{\mathrm{T}} \leq 350~\mathrm{GeV}$, with the
main purpose of establishing an excess above SM. In the absence of a
significant excess, this method provides an inclusive estimate of the
SM backgrounds, including QCD.

One new important development to the method is motivated by the desire
to ensure that the same event topologies and kinematic properties are
observed in the control regions as for the signal region. This is
achieved by scaling all jet kinematic thresholds according to
$H_{\mathrm{T}}^{min} + \ptall$, where $\ptall$ denotes the minimum
\pt of jets considered in the analysis, thus defining the jet
multiplicity (jets falling below this threshold contribute to the \MHT
for the event); and $H_{\mathrm{T}}^{min}$ denotes the lower bound of
the $H_{\mathrm{T}}$ region under consideration. Similarly, the \pt
thresholds of the leading and secondary jets of an event are also
scaled in this way. We define the ratios:
\begin{equation}
x_i = \frac{P^{i}_{\mathrm{T}}}{H_{\mathrm{T}}^{min} + \ptall} 
\end{equation}
where $P^{i}_{\mathrm{T}}$ is the minimum \pt threshold for the
$i^{\mathrm{th}}$ jet. Note that in definition of this analysis,
$P^{i>2}_{\mathrm{T}} \equiv \ptall$. The values of the ratios $x_i$
are determined by the definition of the signal region and kept
constant for the two control regions. The values of the scaled \pt jet
thresholds for the different $H_{\mathrm{T}}$ regions are listed in
Table~\ref{tab:thresholds}. This scaling affects event selection
through all variables that use jets as input (such as the $|\eta|$ of
the leading jet, the jet-outside-acceptance veto, \aT, ...). It should
be noted that this method is based on simple three-body (jet)
kinematics in the transverse plane. In~\cite{cousins}, it was pointed
out that \aT can be expressed in terms of two relative quantities
$x_1$ and $x_2$, which can be identified with the $\rho$ normalisation
parameters in the Dalitz notation.

\medskip
\begin{table}[!h]
 \begin{center}
   \caption{\label{tab:thresholds} Scaled \pt jet thresholds for the
     different $H_{\mathrm{T}}$ regions.}
  \begin{tabular}{cr@{,}c@{,}lcr@{,}c@{,}l}
   \hline\noalign{\smallskip}
   $H_{\mathrm{T}}^{min}$ & 
   ($\ptone$&$\pttwo$&$\ptall$) & 
   $H_{\mathrm{T}}^{min} + \ptall$ &
   ($x_1$&$x_2$&$x_3$) \\
   \noalign{\smallskip}\hline
   250 & (71.4&71.4&35.7) & 285.7 & (0.5&0.5&0.25) \\
   300 & (85.7&85.7&42.9) & 342.9 & (0.5&0.5&0.25) \\
   350 & (100.0&100.0&50.0) & 400.0 & (0.5&0.5&0.25) \\
   \noalign{\smallskip}\hline
  \end{tabular}
 \end{center}
\end{table}

The scaling of the jet $\pt$ thresholds ensures that, when ignoring
instrumental effects, the allowed kinematic phase space is kept
constant for the different $H_{\mathrm{T}}$ regions. Thus, relative
kinematic properties like the jet multiplicity should remain unchanged
between $H_{\mathrm{T}}$ regions.
% and higher multiplicity events should be as abundant in the control regions as in the signal region. 
% This behaviour is important given
% that the relative composition of the individual backgrounds is
% sensitive to the jet multiplicity.

\begin{figure}[!t]
  \begin{center} 
    \subfigure[\label{fig:jet_multiplicity_gen_scaled}
    Monte-Carlo truth and scaled jet \pt thresholds.
    ]{\includegraphics[angle=90,width=0.45\linewidth]{figures/qcd/AN-10-242/GenMultiplicity_Scaled_SM.pdf} }
    \subfigure[\label{fig:jet_multiplicity_gen_fixed}
    Monte-Carlo truth and fixed jet \pt thresholds.
    ]{\includegraphics[angle=90,width=0.45\linewidth]{figures/qcd/AN-10-242/GenMultiplicity_Fixed_SM.pdf} } \\
    \subfigure[\label{fig:jet_multiplicity_scaled}
    Data and scaled jet \pt thresholds.
    ]{\includegraphics[angle=90,width=0.45\linewidth]{figures/qcd/AN-10-242/Multiplicity_Scaled_Data.pdf} }
    \subfigure[\label{fig:jet_multiplicity_fixed}
    Data and fixed jet \pt thresholds.
    ]{\includegraphics[angle=90,width=0.45\linewidth]{figures/qcd/AN-10-242/Multiplicity_Fixed_Data.pdf} }
    \caption{\label{fig:jet_multiplicity} Jet multiplicity
      distributions from Monte-Carlo truth (a and b) and data (c and
      d), normalised to unit area, for the signal and two control
      regions using scaled (a and c) and fixed (b and d) jet \pt
      thresholds. Similar jet multiplicities are observed in each of
      the three $H_{\mathrm{T}}$ regions only when the jet \pt
      thresholds are scaled using $H_{\mathrm{T}}^{min} +
      \ptall$. This is especially true in the absence of instrumental
      effects.}
  \end{center}
\end{figure}

Using jet multiplicity as an example, the necessity of scaling can be
illustrated by considering the \pt thresholds for the leading and
secondary jets in the signal region, for which $H_{\mathrm{T}} >
350~\mathrm{GeV}$ and both jets are required to have $\pt >
100~\mathrm{GeV}$. If these thresholds are also applied in the control
regions within the range $250~\mathrm{GeV} < H_{\mathrm{T}} \leq
350~\mathrm{GeV}$, the available phase space for high-multiplicity
events is restricted with respect to the signal region. 

The validity of this approach is illustrated in
Fig.~\ref{fig:jet_multiplicity_gen_scaled} and
Fig.~\ref{fig:jet_multiplicity_gen_fixed}, which show the jet
multiplicity distributions using Monte-Carlo truth information (\ie,
``gen-jets'') for each of the three $H_{\mathrm{T}}$ regions when the
jet \pt thresholds are scaled and not scaled with
$H_{\mathrm{T}}^{min} + \ptall$, respectively. In the case of the
scaled jet \pt thresholds, the mean jet multiplicity remains constant
within errors; when the jet \pt thresholds are kept fixed for the
three $H_{\mathrm{T}}$ regions, the mean jet multiplicity drops from
$2.6$ for the signal region to $2.0$ for the lower control region,
highlighting the reduced phase space relative to the signal
region. Figs.~\ref{fig:jet_multiplicity_scaled} and
\ref{fig:jet_multiplicity_fixed} show the jet multiplicity
distributions measured from data, again for each of the three
$H_{\mathrm{T}}$ regions when the jet \pt thresholds are scaled and
not scaled with $H_{\mathrm{T}}^{min} + \ptall$, respectively. A
similar behaviour to the gen-jet case is again observed: for fixed jet
\pt thresholds, the mean jet multiplicity drops from $2.8$ for the
signal region to $2.1$ for the lower control region. The (small)
differences with respect to the gen-jet multiplicity distributions are
therefore attributed to instrumental effects. Similar arguments apply
to other kinematic properties and only with the appropriate scaling of
the jet \pt thresholds will it be possible to extract estimators of
kinematic properties from lower $H_{\mathrm{T}}$ regions that can be
applied to the signal region in a controllable fashion.

\subsubsection{$H_{\mathrm{T}}$ scaling when QCD dominated}

Especially important is the application of scaling to the ratio of
events passing and failing a given \aT cut, which is defined as:
\begin{equation}
  \RaT(\theta) = \frac{N(\aT>\theta)}{N(\aT<\theta)}
\end{equation}

In this section of the note, plots showing \RaT as a function
$H_{\mathrm{T}}$ reflect the values of \RaT and $H_{\mathrm{T}}$
calculated when using scaled or fixed jet $p_{\mathrm{T}}$ thresholds,
as indicated on the plots, and are binned according to the value of
$H_{\mathrm{T}}$ (scaled or otherwise). Concerning statistical errors:
the QCD samples were created in \nolinkurl{CMSSW_3_6_X}, which are
limited in size for the $H_{\mathrm{T}}$ regime of the two control
regions, leading to relatively large uncertainties and a normalisation
to $2.2~\mathrm{pb}^{-1}$; in the case of the SM backgrounds with real
\mymet, the samples were normalised to $415~\mathrm{pb}^{-1}$, which
corresponds to the size of the smallest sample available (W+jets).

As a consequence of maintaining the allowed kinematic phase space
constant in each of the three $H_{\mathrm{T}}$ regions, the ratio \RaT
is expected to be independent of the scale $H_{\mathrm{T}}$ for QCD
events in the absence of instrumental effects.

\begin{figure}[!h]
  \begin{center} 
    \subfigure[\label{fig:ratio_vs_ht_051_eq2_gen} Monte-Carlo truth
    information, $N_{\mathrm{jet}} = 2$.
    ]{\includegraphics[angle=90,width=0.4\linewidth]{figures/qcd/AN-10-242/Ratio_Multi2_AlphaT0_51-gen.pdf}}
    \subfigure[\label{fig:ratio_vs_ht_051_ge3_gen} Monte-Carlo truth
    information, $N_{\mathrm{jet}} \geq 3$.
    ]{\includegraphics[angle=90,width=0.4\linewidth]{figures/qcd/AN-10-242/Ratio_Multi3Incl_AlphaT0_51-gen.pdf}} \\
    \subfigure[\label{fig:ratio_vs_ht_051_eq2_scaled} Scaled
    $p_{\mathrm{T}}$ thresholds, $N_{\mathrm{jet}} = 2$.
    ]{\includegraphics[angle=90,width=0.4\linewidth]{figures/qcd/AN-10-242/Ratio_Multi2_AlphaT0_51-baby-scaled.pdf}}
    \subfigure[\label{fig:ratio_vs_ht_051_ge3_scaled} Scaled
    $p_{\mathrm{T}}$ thresholds, $N_{\mathrm{jet}} \geq 3$.
    ]{\includegraphics[angle=90,width=0.4\linewidth]{figures/qcd/AN-10-242/Ratio_Multi3Incl_AlphaT0_51-baby-scaled.pdf}} \\
    \subfigure[\label{fig:ratio_vs_ht_051_eq2_fixed} Fixed
    $p_{\mathrm{T}}$ thresholds, $N_{\mathrm{jet}} = 2$.
    ]{\includegraphics[angle=90,width=0.4\linewidth]{figures/qcd/AN-10-242/Ratio_Multi2_AlphaT0_51-baby-fixed.pdf}}
    \subfigure[\label{fig:ratio_vs_ht_051_ge3_fixed} Fixed
    $p_{\mathrm{T}}$ thresholds, $N_{\mathrm{jet}} \geq 3$.
    ]{\includegraphics[angle=90,width=0.4\linewidth]{figures/qcd/AN-10-242/Ratio_Multi3Incl_AlphaT0_51-baby-fixed.pdf}}
    \caption{\label{fig:ratio_vs_ht_051} The evolution of \RaT as a
      function of $H_{\mathrm{T}}$ for SM backgrounds using
      Monte-Carlo truth (a and b) and reconstructed quantities with
      scaled (c and d) and fixed (e and f) jet $p_{\mathrm{T}}$
      thresholds and for events with $N_{\mathrm{jet}} = 2$ (a, c and
      e) and $N_{\mathrm{jet}} \geq 3$ (b, d and f). The \aT cut value
      used is 0.51. (Markers are artificially offset for clarity.)}
  \end{center}
\end{figure}

\begin{figure}[!h]
  \begin{center} 
    \subfigure[\label{fig:ratio_vs_ht_055_eq2_gen} Monte-Carlo truth
    information, $N_{\mathrm{jet}} = 2$.
    ]{\includegraphics[angle=90,width=0.4\linewidth]{figures/qcd/AN-10-242/Ratio_Multi2_AlphaT0_55-gen.pdf}}
    \subfigure[\label{fig:ratio_vs_ht_055_ge3_gen} Monte-Carlo truth
    information, $N_{\mathrm{jet}} \geq 3$.
    ]{\includegraphics[angle=90,width=0.4\linewidth]{figures/qcd/AN-10-242/Ratio_Multi3Incl_AlphaT0_55-gen.pdf}} \\
    \subfigure[\label{fig:ratio_vs_ht_055_eq2_scaled} Scaled
    $p_{\mathrm{T}}$ thresholds, $N_{\mathrm{jet}} = 2$.
    ]{\includegraphics[angle=90,width=0.4\linewidth]{figures/qcd/AN-10-242/Ratio_Multi2_AlphaT0_55-baby-scaled.pdf}}
    \subfigure[\label{fig:ratio_vs_ht_055_ge3_scaled} Scaled
    $p_{\mathrm{T}}$ thresholds, $N_{\mathrm{jet}} \geq 3$.
    ]{\includegraphics[angle=90,width=0.4\linewidth]{figures/qcd/AN-10-242/Ratio_Multi3Incl_AlphaT0_55-baby-scaled.pdf}} \\
    \subfigure[\label{fig:ratio_vs_ht_055_eq2_fixed} Fixed
    $p_{\mathrm{T}}$ thresholds, $N_{\mathrm{jet}} = 2$.
    ]{\includegraphics[angle=90,width=0.4\linewidth]{figures/qcd/AN-10-242/Ratio_Multi2_AlphaT0_55-baby-fixed.pdf}}
    \subfigure[\label{fig:ratio_vs_ht_055_ge3_fixed} Fixed
    $p_{\mathrm{T}}$ thresholds, $N_{\mathrm{jet}} \geq 3$.
    ]{\includegraphics[angle=90,width=0.4\linewidth]{figures/qcd/AN-10-242/Ratio_Multi3Incl_AlphaT0_55-baby-fixed.pdf}}
    \caption{\label{fig:ratio_vs_ht_055} The evolution of \RaT as a
      function of $H_{\mathrm{T}}$ for SM backgrounds using
      Monte-Carlo truth (a and b) and reconstructed quantities with
      scaled (c and d) and fixed (e and f) jet $p_{\mathrm{T}}$
      thresholds and for events with $N_{\mathrm{jet}} = 2$ (a, c and
      e) and $N_{\mathrm{jet}} \geq 3$ (b, d and f). The \aT cut value
      used is 0.55. (Markers are artificially offset for clarity.)}
  \end{center}
\end{figure}

Figures~\ref{fig:ratio_vs_ht_051_eq2_gen} and
\ref{fig:ratio_vs_ht_051_ge3_gen} show \raT(\aT=0.51) as a function of
$H_{\mathrm{T}}$ for jets clustered using generator information. These
``gen-jets'' do not suffer from instrumental effects and therefore
approximate the environment of an ideal detector.  For the cut value
of $\aT > 0.51$, the ratio \RaT is completely dominated by QCD. For
scaled quantities, both for di-jets and n-jets, the \RaT calculated
from gen-jets is flat within the errors, indicating that the scaling
preserves the kinematic phase space for an ideal detector. However,
for fixed thresholds, already without any detector effects, \RaT
becomes a strong function of $H_{\mathrm{T}}$, indicating that the
kinematic phase space is not preserved. 

In the presence of instrumental effects, \RaT calculated from data
falls monotonically, as can be seen in
Fig.~\ref{fig:ratio_vs_ht_051_eq2_scaled} and
Fig.~\ref{fig:ratio_vs_ht_051_ge3_scaled}. The decrease of \RaT with
$H_{\mathrm{T}}$ is more pronounced in n-jet ($n>2$) events because
for these topologies both jet mis-measurements as well as jets falling
below threshold can cause events to pass the \aT cut. For di-jets only
the effect of jets falling below threshold is typically causing events
to pass the cut and therefore the fall of \RaT with $H_{\mathrm{T}}$
is not as strong as in the n-jet case. Comparable behaviour is
observed for SM backgrounds using a full detector simulation.

In general, the behaviour of \RaT falling with $H_{\mathrm{T}}$ can be
qualitatively understood by assuming that the QCD background has no
significant intrinsic \mymet and only jet mis-measurements lead to the
tail in the \mymet distribution. When considering a measurement of
\RaT using a real detector, the effect of worsening jet energy
resolution with decreasing $H_{\mathrm{T}}$ means that even the scaled
\pt thresholds effectively tighten with decreasing $H_{\mathrm{T}}$,
as jets are more likely to fluctuate below rather than above the jet
\pt threshold. This leads to a decrease in the measured
$H_{\mathrm{T}}$, an increase in \MHT, and thus an increase in the
number of events passing a given \aT cut value, leading to a larger
value of \RaT relative to that for gen-jets (\ie, ideal
detector). This effect is reduced with increasing $H_{\mathrm{T}}$ due
to the improvement in the jet energy resolutions. Hence, \RaT is
expected to be a monotonically falling function of $H_{\mathrm{T}}$
for QCD. This behaviour is observed in data, reported for the
first time in~\cite{SUS-10-001}.
% Importantly, the variable \aT is constructed in such a way that
% mis-measurements essentially always result in a value below 0.5 and
% this falling behaviour is robust against even severe
% mis-measurements, which is an important observation for this method.

For fixed jet $p_{\mathrm{T}}$ thresholds, the same non-scaling
behaviour is also observed in data (and for SM backgrounds using a
full detector simulation, not shown here) as for gen-jets, as
highlighted in
Fig~\ref{fig:ratio_vs_ht_051_eq2_fixed} and
Fig.~\ref{fig:ratio_vs_ht_051_ge3_fixed} for di-jets and
$N_{\mathrm{jet}} \geq 3$, respectively. Therefore, the measured
\RaT($H_{\mathrm{T}}$) with fixed thresholds will be a convolution of
instrumental effects and kinematic phase space changes, which are
almost impossible to separate. On the other hand, for scaled
thresholds the gen-jet ratio stays stable within errors over the
entire range of $H_{\mathrm{T}}$, thus leaving detector effects as the
only source for \RaT($H_{\mathrm{T}}$) to change with
$H_{\mathrm{T}}$.

\subsubsection{$H_{\mathrm{T}}$ scaling when dominated by SM processes
with real \mymet} 

As shown in Sec.~\ref{sec:final_results}, after a cut of $\aT>0.55$
and all other cleaning cuts, only SM backgrounds with real \mymet
(primarily $W^{\pm}$+jets, $Z^{0}\rightarrow\nu\nu$+jets and
$t\bar{t}$) are expected to survive. Therefore, the numerator of \RaT
is completely dominated by events with real \mymet, while the
denominator is effectively QCD
events. Figure~\ref{fig:ratio_vs_ht_055_eq2_gen} and
Fig.~\ref{fig:ratio_vs_ht_055_ge3_gen} show a comparison of "fixed"
and "scaled" \raT(\aT$>0.55$) after all cuts for di-jets and
$N_{\mathrm{jet}} \geq 3$, respectively. In contrast to the same
comparison where QCD was dominating, described in the previous
section, the "fixed" and "scaled" ratio for the sum of all EWK process
are constant within errors. However, when this is split in the
individual process, it becomes apparent that the observed "flatness"
in the case of "fixed" thresholds for $N_{\mathrm{jet}} \geq 3$ is
caused by an accidental compensation of the three individual
background components $W^{\pm}$+jets, $Z^{0}\rightarrow\nu\nu$+jets
and $t\bar{t}$. In particular, the $t\bar{t}$ behaviour displayed in
Fig.~\ref{fig:ratio_vs_ht_055_ge3_ewk_fixed} for $N_{\mathrm{jet}}
\geq 3$ reveals a strong dependence in \RaT, which is explained by the
restriction of phase space caused by the fixed jet \pt thresholds. For
lower $H_{\mathrm{T}}$ bins the available space space is strongly
reduced for $t\bar{t}$ leading to a bias in the \RaT distribution
towards lower values. The vector boson backgrounds, $W^{\pm}$+jets and
$Z^{0}\rightarrow\nu\nu$+jets, exhibit a smaller bias in \RaT but also
suffer from lower statistics due to the fixed thresholds at lower
$H_{\mathrm{T}}$ values, as reflected by the error bars.

For scaled thresholds, however, the phase space is preserved over the
entire range in $H_{\mathrm{T}}$ and therefore no \RaT bias in the
three individual background components is observed. Furthermore, the
scaling of the thresholds also preserves the statistical power of the
control sample at low $H_{\mathrm{T}}$.  This is a clear indication
that also in an EWK-dominated environment, as in the case of
$\aT>0.55$, scaling of cut thresholds is desirable in order to obtain
estimators of kinematic properties in lower $H_{\mathrm{T}}$ bins that
can be applied to the signal region (i.e. higher $H_{\mathrm{T}}$).

\begin{figure}[!h]
  \begin{center} 
    \subfigure[\label{fig:ratio_vs_ht_055_eq2_ewk_scaled}
    Scaled $p_{\mathrm{T}}$ thresholds, $N_{\mathrm{jet}} = 2$.
    ]{\includegraphics[angle=90,width=0.45\linewidth]{figures/qcd/AN-10-242/Ratio_Multi2_AlphaT0_55-ewk-scaled.pdf}} 
    \subfigure[\label{fig:ratio_vs_ht_055_ge3_ewk_scaled}
    Scaled $p_{\mathrm{T}}$ thresholds, $N_{\mathrm{jet}} \geq 3$.
    ]{\includegraphics[angle=90,width=0.45\linewidth]{figures/qcd/AN-10-242/Ratio_Multi3Incl_AlphaT0_55-ewk-scaled.pdf}} \\
    \subfigure[\label{fig:ratio_vs_ht_055_eq2_ewk_fixed}
    Fixed $p_{\mathrm{T}}$ thresholds, $N_{\mathrm{jet}} = 2$.
    ]{\includegraphics[angle=90,width=0.45\linewidth]{figures/qcd/AN-10-242/Ratio_Multi2_AlphaT0_55-ewk-fixed.pdf}}
    \subfigure[\label{fig:ratio_vs_ht_055_ge3_ewk_fixed}
    Fixed $p_{\mathrm{T}}$ thresholds, $N_{\mathrm{jet}} \geq 3$.
    ]{\includegraphics[angle=90,width=0.45\linewidth]{figures/qcd/AN-10-242/Ratio_Multi3Incl_AlphaT0_55-ewk-fixed.pdf}}
    \caption{\label{fig:ratio_vs_ht_055_ewk} The evolution of \RaT as
      a function of $H_{\mathrm{T}}$ for SM backgrounds using
      Monte-Carlo truth, with scaled (a and b) and fixed (c and d) jet
      $p_{\mathrm{T}}$ thresholds and for events with
      $N_{\mathrm{jet}} = 2$ (a and c) and $N_{\mathrm{jet}} \geq 3$
      (b and d). The \aT cut value used is 0.55. (Markers are
      artificially offset for clarity.)}
  \end{center}
\end{figure}

\subsubsection{An empirical $H_{\mathrm{T}}$ scaling law}    

As shown in the previous two sections, after an appropriate scaling of
the jet \pt thresholds, \RaT is either a monotonically falling
function of $H_{\mathrm{T}}$ when QCD dominates (fake \mymet) or constant
with $H_{\mathrm{T}}$ when real \mymet processes are overwhelming.  Since
\RaT is expected to be flat for an ideal detector (i.e. Gen-Jets)
after scaling, the differences between these two scenarios can be
qualitatively understood as the different response of detector effects
for fake and real \mymet signatures. While for fake \mymet topologies
(i.e. QCD events) the resolution and mis-measurements effects
typically improve with energy (i.e. $H_{\mathrm{T}}$), the impact of
mis-measurements on real \mymet signatures for events passing the \aT cut
is small. Therefore, events with significant (real) \mymet are
expected to preserve the behaviour observed for an ideal detector
(i.e. \RaT being flat as a function of $H_{\mathrm{T}}$), whereas for
fake \mymet events \RaT is expected to decrease with increasing
$H_{\mathrm{T}}$.  This qualitative behaviour has been observed in the
Monte Carlo and is also confirmed by the data.

Furthermore, Monte Carlo studies have shown that for both of these
extreme scenarios (fake \mymet dominating and real \mymet dominating)
the $H_{\mathrm{T}}$ behaviour can be well approximated by assuming
the double ratio $R_{\mathrm{R}}$:
\begin{equation}
  R_{\mathrm{R}} = 
  \frac{\RaT(H_{\mathrm{T}})}{\RaT(H_{\mathrm{T}}-X)} =
  \frac{\RaT(H_{\mathrm{T}}+X)}{\RaT(H_{\mathrm{T}})} 
\end{equation}
is constant for three adjacent $H_{\mathrm{T}}$ bins:
$H_{\mathrm{T}}-X, H_{\mathrm{T}}, H_{\mathrm{T}}+X$.

We observe that this ratio is a constant, independent of
$H_{\mathrm{T}}$. In the case of a QCD-dominated sample, the value of
this ratio can be $R_{\mathrm{R}} \ll 1$, whereas for real \mymet
processes, $R_{\mathrm{R}} \approx 1$. While this functional behaviour
can be likely motivated on more general grounds by quantifying the
detector response of real and fake \mymet processes, this goes beyond
the scope of this study.  For the present analysis we take this
``double scaling'' as an empirical fact motivated by the Monte
Carlo. In the following we will demonstrate that this method closes
in the Monte Carlo and that its general behaviour and assumptions are
reproduced in data.  It should be noted that for the analysis only the
real \mymet scenario ($\aT > 0.55$) is of numerical
importance. Therefore, instead of using the double ratio scaling, the
assumption of constant \RaT could have also been a working hypothesis.
However, since the double ratio scaling applies to both scenarios
(\ie, QCD-dominated and EWK-dominated), even though it suffers from
larger statistical uncertainties because \RaT($H_{\mathrm{T}}$) enters
quadratically rather than a constant as in the case of a the "flat"
assumption, the authors decided to adopt this more conservative
extrapolation approach as the baseline. In the following the method is
defined in detail.

\begin{figure}[!t]
  \begin{center} 
%    \subfigure[\label{fig:ratio_vs_ht_low_at}]{\includegraphics[angle=90,width=0.45\linewidth]{figures/qcd/AN-10-242/Ratio_Multi2Incl_AlphaT0_51-bare.pdf}}
%    \subfigure[\label{fig:ratio_vs_ht_high_at}]{\includegraphics[angle=90,width=0.45\linewidth]{figures/qcd/AN-10-242/Ratio_Multi2Incl_AlphaT0_55-bare.pdf}} \\
    \subfigure[\label{fig:ratio_vs_ht_low_at}
    \raT(0.51) as a function of $H_{\mathrm{T}}$.
    ]{\includegraphics[angle=90,width=0.45\linewidth]{figures/qcd/AN-10-242/Ratio_Multi2Incl_AlphaT0_51-baby.pdf}}
    \subfigure[\label{fig:ratio_vs_ht_high_at}
    \raT(0.55) as a function of $H_{\mathrm{T}}$.
    ]{\includegraphics[angle=90,width=0.45\linewidth]{figures/qcd/AN-10-242/Ratio_Multi2Incl_AlphaT0_55-baby.pdf}}
    \caption{\label{fig:ratio_vs_ht} The evolution of the ratio \RaT
      as a function of $H_{\mathrm{T}}$ for events with
      $N_{\mathrm{jet}}\geq2$ and two different \aT cut values: (a)
      0.51 and (b) 0.55. (Markers are artificially offset for
      clarity.)}
  \end{center}
\end{figure}

Applying this independence of the double ratio on $H_{\mathrm{T}}$, we
obtain:
\begin{equation}\label{equ:ratio_predicted}
  \frac{\RaT^{350}}{\RaT^{300}} = 
  \frac{\RaT^{300}}{\RaT^{250}} 
\end{equation}
This equation can thus be solved to yield an expectation for
$\RaT^{350}$ given $\RaT^{250}$ and $\RaT^{300}$:
\begin{equation}\label{equ:ratio_predicted}
  \RaT^{350,\mathrm{pred}} = 
  \frac{\RaT^{300,\mathrm{meas}}}{\RaT^{250,\mathrm{meas}}} 
  \,\cdot\, 
  \RaT^{300,\mathrm{meas}}
\end{equation}
where $\RaT^{350,\mathrm{pred}}$ is the prediction for the signal
region. This ratio is then multiplied with the number of events
observed to fail the \aT cut, $N_{\theta<\aT}^{350,\mathrm{meas}}$ in
the signal region, again measured from data, to obtain an estimate of
the number of events that satisfy $H_{\mathrm{T}} > 350~\mathrm{GeV}$
and $\aT > \theta$:
\begin{equation}\label{equ:number_predicted}
  N_{\theta>\aT}^{350,\mathrm{pred}} = 
  N_{\theta<\aT}^{350,\mathrm{meas}} 
  \,\cdot\, 
  \frac{\RaT^{300,\mathrm{meas}}}{\RaT^{250,\mathrm{meas}}} 
  \,\cdot\, 
  \RaT^{300,\mathrm{meas}}
\end{equation}

Figure~\ref{fig:ratio_vs_ht_low_at} shows the dependence of \RaT on
$H_{\mathrm{T}}$ for data, all SM processes combined and QCD
separately, for events with $N_{\mathrm{jet}}\geq2$ and passing an \aT
cut value of 0.51. 
%The ratio for QCD falls faster than the
% exponential assumption. This is an expected feature, as events
% passing a low \aT cut value are largely due to mis-measurements that
% fall within the Gaussian core of the jet energy resolution
% distribution. This is especially pronounced for the low
% $H_{\mathrm{T}}$ regime, where the resolution is relatively poor,
% which results in a broadening of the \aT edge at 0.5. For higher
% $H_{\mathrm{T}}$ bins, the resolution improves and the edge at 0.5
% sharpens.  \footnote{Note that any contribution to the broadening of
%   the edge at $\alpha_{\mathrm{T}} = 0.5$ due to jets falling below
%   the jet $p_{\mathrm{T}}$ threshold also contribute to the } Thus,
% only severe mis-measurements found within the non-Gaussian tails of
% the resolution distribution and largely attributible to instrumental
% defects can lead to larger \aT values and so a
% faster-than-exponential ``fall-off'' in the ratio \RaT as a function
% of $H_{\mathrm{T}}$ is observed.  This ``fall-off'' effect is less
% pronounced as the \aT cut is tightened.  This ``fall-off'' is less
% pronounced for SM processes with real \mymet, which dominates over
% any contribution from mis-measurements. This differing behaviour for
% QCD and SM can clearly be seen in
% Figure~\ref{fig:ratio_vs_ht_low_at}. Importantly, this feature
% implies that the prediction should be considered as a conservative
% upper bound in the case of QCD and tends towards an estimate for SM
% processes with real \mymet (but is still considered to be an upper
% bound).
For the larger, nominal cut value of 0.55, essentially only severe
mis-measurements result in QCD events passing the \aT cut, even for
the lower $H_{\mathrm{T}}$ bins, leading to very small ratios and a
weak (\ie, essentially flat) dependence on $H_{\mathrm{T}}$, as
expected from EWK processes and as shown in
Fig.~\ref{fig:ratio_vs_ht_high_at}. Essentially no QCD events pass the
\aT cut and only electroweak processes remain.

\subsubsection{Results}

The extrapolation into the signal region yields a predicted number of
events, \Npre, passing the \aT cut, which can be compared with the
number of observed events, \Nobs, in the signal
region. Figure~\ref{fig:diff_vs_at} shows the difference in the number
of events predicted and observed as a function of the \aT cut value,
for data and Monte Carlo simulations of SM processes, with and without
the addition of a SUSY LM1 signal. The cleaning cuts described in
Sec.~\ref{sec:dphistar} and \ref{sec:mhtratio} ($\Delta$R$_{\rm ECAL}$
and $\rmec$) are also applied after the \aT cut.

%The ``fall-off'' behaviour of QCD is manifest as an over-estimate at
%low \aT cut values and the prediction tends to zero with an increasing
%\aT cut value, as QCD is increasingly suppressed by the \aT
%variable. For SM processes, the same behaviour is observed, with the
%difference, $\Npre - \Nobs$, tending towards an estimate of the total
%background with increasing \aT cut values, as QCD is suppressed and
%electroweak processes start to dominate. This trend is also observed
%in data.  Importantly, the difference is always positive or consistent
%with zero, which implies that the method does not under-predict the
%number of events in the signal
%region. Figure~\ref{fig:diff_vs_at_after_baby} also shows $\Npre -
%\Nobs$, but following the \rmec cut, highlighting the power of this
%variable to suppress any remaining QCD background. The contribution
%from QCD is negligible and the background composition is electroweak
%dominated.

For both data and SM processes, the behaviour is flat and consistent
with zero as a function of the \aT cut value. Thus, the predicted and
observed number of events are consistent with one another for all \aT
cut values above 0.55, implying that the prediction provides an
estimate of the total remaining background in the signal
region. Importantly, the difference is always consistent with zero for
the simulation of SM processes, which implies that the method does not
under-predict the number of events in the signal region. The
contribution from QCD is negligible and the background composition is
dominated by processes with real \mymet. Data and SM are in agreement
for all cut values of \aT. Also shown is the behaviour when the SUSY
LM1 signal is added to the SM backgrounds: negative values would be
observed, reflecting an excess above SM expectation.

Table~\ref{tab:total_prediction} summarises the predicted and observed
number of events in the signal region of $H_{\mathrm{T}} >
350~\mathrm{GeV}$ and after an \aT cut of 0.55 for data, QCD, all SM
processes combined and SM+LM1. There is good agreement between data
and SM Monte-Carlo and the QCD contribution is observed to be
negligible.  Table~\ref{tab:numbers_prediction} summarises the values
of the variables used in Equ.~\ref{equ:number_predicted} to give a
prediction for data and SM backgrounds with $\aT > 0.55$

\medskip
\begin{table}[!h]
  \begin{center}
    \caption{\label{tab:total_prediction} Predicted and measured yields
      for data and simulated SM, QCD and LM1 processes for $H_{\mathrm{T}} \geq 350$
      and a cut value $\alpha_{T} = 0.55$. Only statistical uncertainties
      are listed.}
    \begin{tabular}{cr@{$\ \pm\ $}lr@{$\ \pm\ $}lr@{$\ \pm\ $}lr@{$\ \pm\ $}lr@{$\ \pm\ $}lr@{$\ \pm\ $}lr@{$\ \pm\ $}lr@{$\ \pm\ $}l}
      \hline\noalign{\smallskip}
      $N_{\mathrm{jet}}$ &
      \multicolumn{2}{c}{$N^{\mathrm{data}}_{predicted}$} &
      \multicolumn{2}{c}{$N^{\mathrm{data}}_{observed}$} &
      \multicolumn{2}{c}{$N^{\mathrm{SM}}_{predicted}$} & 
      \multicolumn{2}{c}{$N^{\mathrm{SM}}_{observed}$} &
      \multicolumn{2}{c}{$N^{\mathrm{QCD}}_{predicted}$} &
      \multicolumn{2}{c}{$N^{\mathrm{QCD}}_{observed}$} & 
      \multicolumn{2}{c}{$N^{\mathrm{SM+LM1}}_{predicted}$} &  
      \multicolumn{2}{c}{$N^{\mathrm{SM+LM1}}_{observed}$} \\
      \noalign{\smallskip}\hline\noalign{\smallskip}
      2 & $4.9$&$^{4.7}_{3.4}$ & $5$&$2.2$ & $3.1$&$^{3.5}_{2.4}$ & $2.8$&$0.5$ & $0.0$&$^{0.0}_{0.0}$ & $0.0$&$0.0$ & $4.7$&$^{4.2}_{2.9}$ & $9.8$&$0.5$ \\
      $\geq$3 & $5.2$&$^{3.4}_{2.6}$ & $9$&$3$ & $6.1$&$^{4.5}_{3.4}$ & $6.4$&$0.7$ & $0.0$&$^{0.0}_{0.0}$ & $0.1$&$0.0$ & $7.2$&$^{5.3}_{4.1}$ & $18.6$&$0.7$ \\
      $\geq$2 & $9.0$&$^{4.5}_{3.8}$ & $14$&$3.7$ & $9.1$&$^{5.4}_{4.3}$ & $9.2$&$0.9$ & $0.0$&$^{0.0}_{0.0}$ & $0.1$&$0$ & $11.8$&$^{6.4}_{5.2}$ & $28.5$&$0.9$ \\
      \noalign{\smallskip}\hline
    \end{tabular}
  \end{center}
\end{table}

\medskip
\begin{table}[!h]
  \begin{center}
    \caption{\label{tab:numbers_prediction} Values of the variables used
      in Equ.~\ref{equ:number_predicted} to give a prediction for data
      and SM backgrounds with $\aT > 0.55$. Only statistical
      uncertainties are listed.}
    \begin{tabular}{cr@{$\ \pm\ $}lr@{$\ \pm\ $}lr@{$\ \pm\ $}lr@{$\ \pm\ $}lr@{$\ \pm\ $}lr@{$\ \pm\ $}l}
      \hline\noalign{\smallskip}
      $N_{\mathrm{jet}}$ &
      \multicolumn{2}{c}{$R^{250,\mathrm{data}}_{\aT}$} &
      \multicolumn{2}{c}{$R^{300,\mathrm{data}}_{\aT}$} &
      \multicolumn{2}{c}{$N^{350,\mathrm{data}}_{\aT<\theta}$} &
      \multicolumn{2}{c}{$R^{250,\mathrm{SM}}_{\aT}$} &
      \multicolumn{2}{c}{$R^{300,\mathrm{SM}}_{\aT}$} &
      \multicolumn{2}{c}{$N^{350,\mathrm{SM}}_{\aT<\theta}$} \\
      &
      \multicolumn{2}{c}{[$10^{-6}$]} &
      \multicolumn{2}{c}{[$10^{-6}$]} &
      \multicolumn{2}{c}{} &
      \multicolumn{2}{c}{[$10^{-6}$]} &
      \multicolumn{2}{c}{[$10^{-6}$]} &
      \multicolumn{2}{c}{} \\
      \noalign{\smallskip}\hline\noalign{\smallskip}
      $\geq$2 & 
      $41.1$&$^{7.4}_{6.6}$ & 
      $33.1$&$^{11.1}_{9.1}$ & 
      $336063$&$580$ & 
      $24.7$&$^{6.0}_{5.1}$ & 
      $26.8$&$^{10.1}_{8.0}$ & 
      $315312$&$562$ \\
      \noalign{\smallskip}\hline
    \end{tabular}
  \end{center}
\end{table}

\subsubsection{Systematic studies}

As shown already, when scaling the jet \pt thresholds, the kinematic
properties of the signal region are to a good approximation preserved in
the lower $H_{\mathrm{T}}$ control regions.  This largely negates any
contribution to systematic uncertainty due to differing kinematics in
each of the control regions. Also, any systematic effects, such as jet
mis-measurement or identification, that may occur in the signal region
will also affect the control regions. Thus, to a good approximation,
these systematic effects are already accounted for in the
method. Therefore, in the absences of significant QCD contributions to
the ratio, the $H_{\mathrm{T}}$ scaling method is expected to provide an estimate
of the remaining SM backgrounds with real \mymet. In the following
sections we discuss additional cross checks and systematic studies in
support of this expectation.

\subsubsection{Closure in the Monte Carlo and comparison with data}
\begin{figure}[!t]
  \begin{center} 
    \subfigure[\label{fig:diff_vs_at_mc}
    Monte-Carlo closure test.
    ]{\includegraphics[angle=90,width=0.45\linewidth]{figures/qcd/AN-10-242/DifferenceVsAlphaT_Multi-2_mc-baby.pdf} }
    \subfigure[\label{fig:diff_vs_at_data}
    Comparison of data and simulation.
    ]{\includegraphics[angle=90,width=0.45\linewidth]{figures/qcd/AN-10-242/DifferenceVsAlphaT_Multi-2_data-baby.pdf} }
    \caption{\label{fig:diff_vs_at} The difference in the number of
      events predicted by the method and observed as a function of the
      \aT cut value. (a) A Monte-Carlo closure test, showing expected
      behaviour for SM processes, with and without the addition of a SUSY LM1
      signal. (b) Comparison of data with SM expectation, as also
      shown in Fig (a). (Markers are artificially offset for clarity.)}
%    \caption{\label{fig:diff_vs_at} The difference in the number of
%      events predicted by the method and observed as a function of the
%      \aT cut value, for data and Monte Carlo simulations of QCD and
%      SM (\ie, QCD and electroweak) processes, following (a) the \aT
%      cut and (b) additionally the \rmec cut.}
  \end{center}
\end{figure}

Using a full Monte-Carlo detector simulation, a consistent closure
test was carried out in order to establish that the method provides an
unbiased estimator of the number of total SM background events
surviving all in cuts in the signal region. The result of this test is
displayed in Fig.~\ref{fig:diff_vs_at_mc}, which shows the difference
between the number of predicted and observed events as a function of
the \aT cut value. The Monte Carlo closure test is within errors
distributed around $\Npre - \Nobs = 0$, demonstrating that at least
for the Monte Carlo the method provides a good estimator of the
measured number of background events after all cuts.  When adding
signal (LM1) to Monte Carlo sample, a clear under prediction is
observed showing that in the presence of significant signal
contamination this method would be a good indicator to establish a
deviation from the SM hypothesis without first measuring the
individual background components. As can be seen from
Fig.~\ref{fig:diff_vs_at_data}, $\Npre - \Nobs$ measured in data is
compatible with zero and therefore indicates that no significant
deviation from the SM hypothesis is observed.

\subsubsection{Variation of EWK background in the Monte Carlo}

\begin{figure}[!t]
  \begin{center} 
    \subfigure[\label{fig:ratio_vs_ht_high_at_ewk} 
    %The behaviour of \RaT as a function of $H_{\mathrm{T}}$ for each
    %of the individual background processes $W^{\pm}$+jets,
    %$Z^{0}\rightarrow\nu\nu$, $t\bar{t}$ and their total, in the
    %absence of QCD events passing the \aT cut.
    \RaT versus $H_{\mathrm{T}}$ for SM processes with real \mymet.
    ]{\includegraphics[angle=90,width=0.45\linewidth]{figures/qcd/AN-10-242/Ratio_Multi2Incl_AlphaT0_55-ewk.pdf}}
    \subfigure[\label{fig:ratio_vs_ht_high_at_ttbar} 
    %Dependence of \RaT on $H_{\mathrm{T}}$ for the sum of the three
    %processes above and the effect of varying the $t\bar{t}$ component
    %only, after an \aT cut value of 0.55.
    \RaT versus $H_{\mathrm{T}}$ when varying $t\bar{t}$ contribution.
    ]{\includegraphics[angle=90,width=0.45\linewidth]{figures/qcd/AN-10-242/Ratio_Multi2Incl_AlphaT0_55-ttbar.pdf}}
    \caption{\label{fig:ratio_vs_ht_ewk} (a) The behaviour of \RaT as
      a function of $H_{\mathrm{T}}$ for each of the individual
      background processes $W^{\pm}$+jets, $Z^{0}\rightarrow\nu\nu$+jets,
      $t\bar{t}$ and their total, in the absence of QCD events passing
      the \aT cut. (b) Dependence of \RaT on $H_{\mathrm{T}}$ for the
      sum of the three processes above and the effect of varying the
      $t\bar{t}$ component only, after an \aT cut value of 0.55.
      %Behaviour of \RaT for SM processes with real \mymet. 
      (Markers are artificially offset for clarity.)}
  \end{center}
\end{figure}

Although the method closes in the Monte Carlo it is important to check
the assumption of the double ratio scaling behaviour of \RaT as a
function of $H_{\mathrm{T}}$, \ie the independence of \RaT on
$H_{\mathrm{T}}$, which is used when extrapolating into the signal
region. After all cuts, the numerator of \RaT is dominated by EWK
processes with real \mymet.  Figure~\ref{fig:ratio_vs_ht_high_at_ewk}
shows the behaviour of \RaT as a function of $H_{\mathrm{T}}$ for each
of the individual background processes ($W^{\pm}$+jets,
$Z^{0}\rightarrow\nu\nu$+jets, $t\bar{t}$) and their total, in the
absence of QCD events passing the \aT cut. All the plots have been
normalised to the same effective cross-section in the signal region,
so that the shapes can be compared easily. The shapes are essentially
flat and the ratio, \RaT, in the $250~\mathrm{GeV} < H_{\mathrm{T}}
\leq 300~\mathrm{GeV}$ bin differs at most by $20\%$ relative to the
total.

Given the very weak dependence of \RaT on $H_{\mathrm{T}}$, a stress
test was devised to see the effect of drastically varying the
composition of the SM background. This was done by effectively scaling
the cross-sections of each of the $W^{\pm}$+jets,
$Z^{0}\rightarrow\nu\nu$+jets and $t\bar{t}$ processes in
turn.\footnote{QCD was not considered here, as the number of events
  passing $\aT > 0.55$ is negligible.} The cross-sections of each of
the backgrounds under consideration were varied very conservatively by
$\pm100\%$, $+200\%$, $+300\%$, $+400\%$ and $+500\%$. The
extrapolation into the signal region was then repeated for each of the
various mixtures. Figure~\ref{fig:ratio_vs_ht_high_at_ttbar} shows the
dependence of \RaT on $H_{\mathrm{T}}$ for the sum of the three
processes above and the effect of varying the $t\bar{t}$ component
only, after an \aT cut value of 0.55. While the absolute values of
\RaT change by as much as $\sim250\%$ relative to the nominal value,
the shape of the \RaT distribution versus $H_{\mathrm{T}}$ is largely
unaffected. The same behaviour is observed when varying the
$W^{\pm}$+jets and $Z^{0}\rightarrow\nu\nu$+jets processes. This is
highlighted by Fig.~\ref{fig:prediction_diff}, which shows the
difference in the number of events predicted for each different
background composition and the corresponding number of events observed
in the signal region, $\Npre - \Nobs$. The spread of the distribution
is $11\%$ and the bias of $1\%$ is consistent with zero when
considering the statical uncertainties associated with the predicted
and observed numbers.

Therefore, even after varying drastically the composition of the
individual EWK components, no significant bias in the number of
predicted events is observed in the Monte Carlo, suggesting that the
method is very robust against the composition of the total EWK
background.

\begin{figure}[!t]
  \begin{center}
    \includegraphics[angle=90,width=0.45\textwidth]{figures/qcd/AN-10-242/Difference_AlphaT_550_Multi-2_baby.pdf}
    \caption{\label{fig:prediction_diff} The difference in the number
      of predicted and observed events, $\Npre - \Nobs$, for different
      mixtures of SM backgrounds. }
  \end{center} 
\end{figure}



\subsubsection{Validation of the constant $R_{\mathrm{R}}$
  expectation, when dominated by SM processes with real \mymet, with
  data}

The observed constant behaviour of \RaT with $H_{\mathrm{T}}$ can also
be validated in data using the EWK estimate defined in section
Sec.~\ref{sec:inclmuon}. This method uses a muon control sample
measured in data to predict the remaining background of
$W^{\pm}$+jets, $Z^{0}\rightarrow\nu\nu$+jets and $t\bar{t}$
(Table~\ref{tab:muonEventPrediction}).  Based on these estimates, for
these processes with real \mymet, the ratio \RaT can be calculated by
dividing the EWK estimates for the different $H_{\mathrm{T}}$ bins by
the corresponding number of events failing all cuts.
Table~\ref{tab:muon_validation} lists the number of EWK events
predicted by the muon control sample to survive all cuts, the number
of events measured in data failing all cuts, as well as the
corresponding ratio, $R^{\mathrm{data}}_{\mathrm{all\, cuts}}$, for
the three $H_{\mathrm{T}}$ bins. Also listed are the predictions of
the ratio, $R^{\mathrm{SM}}_{\mathrm{all\, cuts}}$, from Monte-Carlo
for the remaining EWK background.

While overall the ratios measured in data are approximately 30\%
higher than in the Monte Carlo, the ratio is constant with $H_{\mathrm{T}}$
within errors and therefore confirms the expectation from Monte Carlo
that \RaT is flat as a function of $H_{\mathrm{T}}$ after all
cuts. Therefore, the assumed $H_{\mathrm{T}}$ scaling behaviour is
expected to predict an unbiased estimator of the \RaT in the signal
regions when SM processes with real \mymet dominate the numerator of
the ratio.

\medskip
\begin{table}[!h]
  \begin{center}
    \caption{\label{tab:muon_validation} Validation of the scaling law
      with measurements from data of \RaT in the control regions using
      a muon control sample.}
    \begin{tabular}{cr@{$\ \pm\ $}lr@{$\ \pm\ $}lr@{$\ \pm\ $}lr@{$\ \pm\ $}lr@{$\ \pm\ $}lr@{$\ \pm\ $}l}
      \hline\noalign{\smallskip}
      $H_{\mathrm{T}}$ &
      \multicolumn{2}{c}{$N^{\mathrm{data}}_{\mathrm{all\, cuts}}$} &
      \multicolumn{2}{c}{$N^{\mathrm{data}}_{\mathrm{all\, cuts}}$} &
      \multicolumn{2}{c}{$R^{\mathrm{data}}_{\mathrm{all\, cuts}}$} &
      \multicolumn{2}{c}{$N^{\mathrm{MC}}_{\mathrm{all\, cuts}}$} &
      \multicolumn{2}{c}{$N^{\mathrm{MC}}_{\mathrm{all\, cuts}}$} &
      \multicolumn{2}{c}{$R^{\mathrm{MC}}_{\mathrm{all\, cuts}}$} \\
      &
      \multicolumn{2}{c}{} &
      \multicolumn{2}{c}{} &
      \multicolumn{2}{c}{[$10^{-5}$]} &
      \multicolumn{2}{c}{} &
      \multicolumn{2}{c}{} &
      \multicolumn{2}{c}{[$10^{-5}$]} \\
      \noalign{\smallskip}\hline\noalign{\smallskip}
      250 & 
      29.1&$^{6.7}_{8.7}$ & 
      851177&923 & 
      3.9&$^{0.9}_{1.2}$ & 
      16.9&1.2 &
      810613&900 &
      2.4&$^{0.6}_{0.5}$ \\
      300 &
      9.3&$^{5.2}_{3.3}$ & 
      332284&576 &
      3.2&$^{1.8}_{1.1}$ & 
      7.0&0.8 &
      316844&563 &
      2.6&$^{1.0}_{0.8}$ \\
      350 &
      11.0&$^{6.0}_{4.8}$ &
      336063&580 &
      3.3&$^{1.8}_{1.4}$  &
      7.5&0.8 &
      315312&562 &
      2.4&$^{1.1}_{0.9}$ \\
      \noalign{\smallskip}\hline
    \end{tabular}
  \end{center}
\end{table}



\subsubsection{Cross-checks using alternative methods}

We present here two alternative methods for predicting the SM
backgrounds in the signal region as an independent cross-check to the
double ratio scaling.

In the first case, the background prediction is based on the measurement of
$\RaT^{300}$ from data in the control region $300~\mathrm{GeV} <
H_{\mathrm{T}} \leq 350~\mathrm{GeV}$. A small correction from Monte
Carlo is then used to extrapolate into the signal region:
\begin{equation}
  \RaT^{350,\mathrm{pred}} =
  \frac{\RaT^{350,\mathrm{MC}}}{\RaT^{300,\mathrm{MC}}} 
  \,\cdot\, 
  \RaT^{300,\mathrm{meas}} 
\end{equation}
where $\RaT^{350,\mathrm{pred}}$ is the prediction for the signal
region. No assumption on the functional form of \RaT is made in this
case and the method relies purely on Monte Carlo to extrapolate to the
signal region. Again, the ratio then multiplied with the number of
events observed to fail the \aT cut in the signal region,
$N_{\theta<\aT}^{350,\mathrm{meas}}$, to obtain an estimate of the
number of events from SM backgrounds that satisfy $\aT > \theta$:
\begin{equation}
  N_{\theta>\aT}^{350,\mathrm{pred}} = 
  N_{\theta<\aT}^{350,\mathrm{meas}}
  \,\cdot\, 
  \frac{\RaT^{350,\mathrm{MC}}}{\RaT^{300,\mathrm{MC}}} 
  \,\cdot\, 
  \RaT^{300,\mathrm{meas}}
\end{equation}

Figure~\ref{fig:diff_vs_at_after_method3} shows the difference between
the number of events predicted and observed in the signal region as a
function of the \aT cut value, for data and also for the sum of SM
processes and a SUSY LM1 signal.\footnote{Again, the cleaning cuts
  $\Delta$R$_{\rm ECAL}$ and $\rmec$ are applied after the \aT cut.}
The prediction for data is again consistent with the observation for
all cut values of \aT and the expected behaviour is observed when the
SUSY LM1 signal is added to the SM backgrounds: negative values
reflecting an excess above SM expectation.

For both data and SM+LM1, the MC-based correction used to extrapolate
into the signal region encompasses all the SM processes. However, as
shown in the previous sections, the contribution from QCD is
negligible and the behaviour of \RaT versus $H_{\mathrm{T}}$ for the
remaining processes (with real \mymet) is essentially flat for an \aT
cut of 0.55. Thus, the correction factor, $R^{\mathrm{SM}}_{350/300}$,
is close to $1$, as shown in
Table~\ref{tab:total_prediction_method3}. For comparison purposes
only, the same ratios measured from data and for SM+LM1 are also
shown. Again, there is good agreement between predicted and observed
numbers of events in the signal region for data.

\begin{figure}[!t]
  \begin{center} 
    \subfigure[\label{fig:diff_vs_at_after_method3}
    ]{\includegraphics[width=0.45\linewidth]{figures/qcd/AN-10-242/DifferenceVsAlphaT_Multi-2_method3.pdf}}
    \subfigure[\label{fig:prediction_diff_method3}
    ]{\includegraphics[angle=90,width=0.45\linewidth]{figures/qcd/AN-10-242/Difference_AlphaT_550_Multi-2_method3.pdf}}
    \caption{\label{fig:diff_vs_at_method3} (a) The difference in the
      number of events predicted and observed for data and SM+LM1, as
      a function of the \aT cut value. (b) The difference in the
      number of predicted and observed events for different background
      compositions.}
  \end{center}
\end{figure}

\medskip
\begin{table}[!h]
 \begin{center}
   \caption{\label{tab:total_prediction_method3} Predicted and measured yields
     for data and simulated SM, QCD and LM1 processes for $H_{\mathrm{T}} \geq 350$
     and a cut value $\alpha_{T} = 0.55$. Only statistical uncertainties
     are listed.}
  \begin{tabular}{cr@{$\ \pm\ $}lr@{$\ \pm\ $}lr@{$\ \pm\ $}lr@{$\ \pm\ $}lr@{$\ \pm\ $}lr@{$\ \pm\ $}lr@{$\ \pm\ $}l}
   \hline\noalign{\smallskip}
   $N_{\mathrm{jet}}$ & 
   \multicolumn{2}{c}{$R^{\mathrm{SM}}_{350/300}$} &
   \multicolumn{2}{c}{$R^{\mathrm{data}}_{350/300}$} &
   \multicolumn{2}{c}{$R^{\mathrm{SM+LM1}}_{350/300}$} &
   \multicolumn{2}{c}{$N^{\mathrm{data}}_{predicted}$} &
   \multicolumn{2}{c}{$N^{\mathrm{data}}_{observed}$} &
   \multicolumn{2}{c}{$N^{\mathrm{SM+LM1}}_{predicted}$} &
   \multicolumn{2}{c}{$N^{\mathrm{SM+LM1}}_{observed}$} \\
   \noalign{\smallskip}\hline\noalign{\smallskip}
   2 & 
   $1.1$&$^{0.3}_{0.3}$ & 
   $1.5$&$^{1.2}_{0.9}$ &
   $3.2$&$^{2.1}_{1.5}$ & 
   $3.8$&$^{2.5}_{1.9}$ & 
   $5$&$2.2$ &
   $3.5$&$^{2.2}_{1.5}$ & 
   $9.8$&$0.5$ \\
   $\geq$3 & 
   $1.0$&$^{0.2}_{0.2}$ &
   $1.1$&$^{0.6}_{0.5}$ & 
   $2.7$&$^{1.5}_{1.1}$ &
   $8.5$&$^{4.0}_{3.2}$ & 
   $9$&$3$ &
   $7.1$&$^{3.6}_{2.8}$ & 
   $18.6$&$0.7$ \\
   $\geq$2 &
   $1.1$&$^{0.2}_{0.1}$ &
   $1.3$&$^{0.6}_{0.5}$ & 
   $2.9$&$^{1.1}_{0.9}$ &
   $12.0$&$^{4.4}_{3.7}$ & 
   $14$&$3.7$ &
   $10.6$&$^{4.0}_{3.2}$ &
   $28.5$&$0.9$ \\
   \noalign{\smallskip}\hline
  \end{tabular}
 \end{center}
\end{table}

\begin{figure}[!h]
  \begin{center} 
    \includegraphics[width=0.45\linewidth]{figures/qcd/AN-10-242/Ratio_Multi2Incl_AlphaT0_55-with-fit.pdf}
    \caption{\label{fig:ratio_with_fit} The behaviour of \RaT as a
      function of HT with scaled jet \pt thresholds and an \aT cut
      value of 0.55 for events with $N_{\mathrm{jet}} \geq 2$. The
      five bins within $250~\mathrm{GeV} \geq HT < 350~\mathrm{GeV}$
      are used by a fitting procedure that assumes flat
      behaviour. (Markers are artificially offset for clarity.)}
  \end{center}
\end{figure}

\medskip
\begin{table}[!h]
  \begin{center}
    \caption{\label{tab:check_prediction} Values of the variables used
      in Equ.~\ref{equ:number_predicted} to give a prediction for data
      and SM backgrounds with $\aT > 0.55$. Only statistical
      uncertainties are listed.}
    \begin{tabular}{cr@{$\ \pm\ $}lr@{$\ \pm\ $}lr@{$\ \pm\ $}lr@{$\ \pm\ $}l}
      \hline\noalign{\smallskip}
      Sample &
      \multicolumn{2}{c}{$R^{\mathrm{fit},\mathrm{meas}}_{\aT}$} &
      \multicolumn{2}{c}{$N^{350,\mathrm{meas}}_{\aT<\theta}$} &
      \multicolumn{2}{c}{$N^{350}_{\mathrm{predicted}}$} &
      \multicolumn{2}{c}{$N^{350}_{\mathrm{observed}}$} \\
      &
      \multicolumn{2}{c}{[$10^{-5}$]} &
      \multicolumn{2}{c}{} &
      \multicolumn{2}{c}{} &
      \multicolumn{2}{c}{} \\
      \noalign{\smallskip}\hline\noalign{\smallskip}
      Data & 
      3.4&0.6 & 
      336063&580 & 
      11.4&2.0 &
      14&3.7 \\
      SM &
      2.5&0.1 &
      315312&562 &
      7.9&0.4 &
      9.2&0.9 \\
      \noalign{\smallskip}\hline
    \end{tabular}
  \end{center}
\end{table}

The correction taken from Monte-Carlo has a systematic uncertainty
associated with it, which can be estimated by repeating the same
method reported earlier of varying the effective cross-section of the
individual background components by $\pm100\%$, $+200\%$, $+300\%$,
$+400\%$ and $+500\%$. Figure~\ref{fig:prediction_diff_method3} shows
the difference in the number of events predicted for each different
background composition and the corresponding number of events observed
in the signal region, $\Npre - \Nobs$. The spread of the distribution
is $3\%$ and the bias is negligible.

%\begin{table}[!h]
% \begin{center}
%  \begin{tabular}{cr@{$\ \pm\ $}lr@{$\ \pm\ $}lr@{$\ \pm\ $}lr@{$\ \pm\ $}lr@{$\ \pm\ $}lr@{$\ \pm\ $}lr@{$\ \pm\ $}l}
%   \hline\noalign{\smallskip}
%   $N_{\mathrm{jet}}$ & 
%   \multicolumn{2}{c}{$R^{\mathrm{SM}}_{350/250}$} &
%   \multicolumn{2}{c}{$R^{\mathrm{data}}_{350/250}$} &
%   \multicolumn{2}{c}{$R^{\mathrm{SM+LM1}}_{350/250}$} &
%   \multicolumn{2}{c}{$N^{\mathrm{data}}_{predicted}$} &
%   \multicolumn{2}{c}{$N^{\mathrm{data}}_{observed}$} &
%   \multicolumn{2}{c}{$N^{\mathrm{SM+LM1}}_{predicted}$} &
%   \multicolumn{2}{c}{$N^{\mathrm{SM+LM1}}_{observed}$} \\
%   \noalign{\smallskip}\hline\noalign{\smallskip}
%   2 & 
%   $1.5$&$^{1.2}_{0.8}$ & 
%   $2.2$&$^{1.5}_{1.1}$ & 
%   $4.9$&$^{1.5}_{1.3}$ & 
%   $3.4$&$^{3.1}_{2.1}$ & $5$&$2.2$ &
%   $8.7$&$^{7.3}_{5.0}$ & $28.3$&$1.5$ \\
%   $\geq$3 & 
%   $1.0$&$^{0.5}_{0.4}$ & 
%   $0.7$&$^{0.3}_{0.2}$ & 
%   $2.9$&$^{0.6}_{0.6}$ &
%   $13.0$&$^{7.4}_{5.9}$ & $9$&$3$ &
%   $18.9$&$^{10.4}_{8.3}$ & $53.7$&$2.0$ \\
%   $\geq$2 & 
%   $1.2$&$^{0.5}_{0.4}$ & 
%   $1.0$&$^{0.3}_{0.3}$ & 
%   $3.5$&$^{0.6}_{0.6}$ &
%   $16.2$&$^{7.7}_{6.4}$ & $14$&$3.7$ &
%   $27.5$&$^{12.6}_{10.4}$ & $82.0$&$2.5$ \\
%   \noalign{\smallskip}\hline
%  \end{tabular}
%  \caption{method3, lm1, ht=250}
% \end{center}
%\end{table}

A further cross-check can be made by assuming a flat behaviour for
\RaT as a function of $H_{\mathrm{T}}$ and fitting in the control
regions to predict the number of events in the signal
region. Figure~\ref{fig:ratio_with_fit} shows the behaviour of \RaT as
a function of $H_{\mathrm{T}}$ with scaled jet $p_{\mathrm{T}}$
thresholds and an \aT cut value of 0.55 for events with
$N_{\mathrm{jet}} \geq 2$ (\ie, the fully inclusive sample, which
corresponds to the sum of Fig.~\ref{fig:ratio_vs_ht_055_eq2_scaled}
and Fig.~\ref{fig:ratio_vs_ht_055_ge3_scaled}). The five bins within
$250~\mathrm{GeV} \leq H_{\mathrm{T}} < 350~\mathrm{GeV}$ are used by a
fitting procedure that assumes flat behaviour. The fit results are
summarized in Table~\ref{tab:check_prediction}, along with a
prediction for the signal region, given by multiplying the value from
the fit by the number of events failing the \aT cut in the signal
region, $N^{350,\mathrm{meas}}_{\aT<\theta}$, measured from
data. Again, the prediction agrees well with observed.

\subsubsection{Summary}

Using this data-driven method, we predict the total number of
background events in the signal region defined by $H_{\mathrm{T}} >
350~\mathrm{GeV}$ and $\aT > 0.55$ to be $9.0\pm^{4.5}_{3.8}$ ({\it
  stat}). Various cross-checks and systematic studies have been
performed without revealing any significant bias in the measurement,
primarily due to its data-driven nature. For that reason, we do not
assign a systematic uncertainty to this measurement.

A second independent method, that uses a small correction from Monte
Carlo, provides a cross-check that gives $12.0\pm^{4.4}_{3.7}$ ({\it
  stat}) $\pm 0.4$ ({\it sys}). A third method, based on the
assumption of \RaT being independent of $H_{\mathrm{T}}$ (\ie, flat)
when being EWK-dominated gives $11.4\pm2.0$ ({\it stat}).

All methods are based on scaling the jet \pt thresholds in the lower
$H_{\mathrm{T}}$ control regions and their results agree within the
errors.  Furthermore, all methods have been shown in the MC to provide
an unbiased estimator of the number of total background events after
all cuts. 

For the final result, we choose the first method, which relies
entirely on properties measured in the data (\ie, the ratios), and
describes both the QCD-dominated and EWK-dominated $H_{\mathrm{T}}$
scaling behaviours. Alternatively, we could have also adopted a more
aggressive approach that explicitly relies on the assumption of
flat behaviour of \RaT for the EWK-dominated environment when
extrapolating into the signal region.

